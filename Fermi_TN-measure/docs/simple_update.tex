\documentclass[11pt]{article}
\usepackage[a4paper, margin=2cm]{geometry}
\usepackage{amsmath}
\usepackage{amssymb}
\usepackage{amsthm}
\usepackage{dsfont}
\usepackage{bbm}
\usepackage{multirow}
\usepackage{array}
\usepackage{diagbox}
\usepackage{makecell}
\newcolumntype{C}[1]{>{\centering\arraybackslash}m{#1}}
% \usepackage{charter}
\usepackage{fontspec}
% \usepackage{unicode-math}
\setmainfont{Charter}
% \setmathfont{XCharter-Math.otf}
% \usepackage{newtxtext,newtxmath}
\usepackage{braket}
\usepackage{slashed}
\usepackage{color}
\usepackage[table]{xcolor}
\usepackage[mathscr]{euscript}
\usepackage{graphicx}
\usepackage{placeins}
\usepackage{floatrow}
\floatsetup[table]{capposition=top}
\usepackage[caption=false]{subfig}
\usepackage[export]{adjustbox}
\floatsetup[figure]{style=plain,subcapbesideposition=top}
\usepackage{xfrac}
\usepackage{bm}
\usepackage{microtype}
\usepackage{commath}
\usepackage{mathtools}
\usepackage{enumitem}
\usepackage{xparse}
\usepackage[
    colorlinks=true, urlcolor=blue,
    linkcolor=blue, citecolor=blue
]{hyperref}% add hypertext capabilities
\usepackage{tikz}
\usepackage{ifthen}
\usepackage[framemethod=TikZ]{mdframed}
\usepackage[version=4]{mhchem}
\usepackage[numbers]{natbib}
\usepackage[nottoc]{tocbibind}
\usepackage[english]{babel}
\usepackage[autostyle, english = american]{csquotes}
\MakeOuterQuote{"}
\usepackage{setspace}

\usetikzlibrary{decorations.markings}
\usetikzlibrary{calc, positioning, arrows.meta}
\tikzset{midarrow/.style={
    decoration={markings, mark=at position 0.55 with {\arrow{>}}},
    postaction={decorate}
}}
\tikzset{midarrowrev/.style={
    decoration={markings, mark=at position 0.45 with {\arrow{<}}},
    postaction={decorate}
}}

% \allowdisplaybreaks

\theoremstyle{remark}
\newtheorem*{remark}{Remark}

\NewDocumentEnvironment{diagram}{O{0.68} O{0.75}}{
    \begin{tikzpicture}[
        baseline = (X.base),
        every node/.style={scale=#1}, scale=#2
    ]
}{\end{tikzpicture}}

\definecolor{green}{RGB}{50, 180, 50}
\definecolor{blue}{RGB}{20, 30, 250}

\newcommand{\pos}[2]{\begin{matrix}
    #1 \\ #2
\end{matrix}}
\newcommand{\dobase}[2]{
    \draw (#1,#2) node (X) {$\phantom{X}$};
}

\NewDocumentCommand{\lineH}{O{black} m m m}{
    \draw[color=#1] (#2,#4) -- (#3,#4);
}
\NewDocumentCommand{\lineHa}{O{black} m m m}{
    \draw[midarrow, color=#1] (#2,#4) -- (#3,#4);
}

\NewDocumentCommand{\lineV}{O{black} m m m}{
    \draw[color=#1] (#4,#2) -- (#4,#3);
}
\NewDocumentCommand{\lineVa}{O{black} m m m}{
    \draw[midarrow, color=#1] (#4,#2) -- (#4,#3);
}

\newcommand{\contrL}[3]{
    \draw (#3,#1) to[out=180,in=180] (#3,#2);
}
\newcommand{\contrLa}[3]{
    \draw[midarrow] (#3,#1) to[out=180,in=180] (#3,#2);
}

\newcommand{\contrR}[3]{
    \draw (#3,#1) to[out=0,in=0] (#3,#2);
}
\newcommand{\contrRa}[3]{
    \draw[midarrow] (#3,#1) to[out=0,in=0] (#3,#2);
}

\NewDocumentCommand{\rect}{m m m m m o}{
    \IfNoValueTF{#6}{
        \draw[rounded corners] 
        (#1 - #3/2, #2 - #4/2) 
        rectangle (#1 + #3/2, #2 + #4/2);
    }{
        \draw[rounded corners, fill=#6] 
        (#1 - #3/2, #2 - #4/2) 
        rectangle (#1 + #3/2, #2 + #4/2);
    }
    \node[align=center] at (#1,#2) {#5};
}

\newcommand{\tensor}[3]{
    \rect{#1}{#2}{1}{1}{#3}
}
\newcommand{\Tensora}[1]{
    \begin{diagram}
        \dobase{0}{0} \tensor{0}{0}{#1}
        \lineHa{-1}{-0.5}{0}
        \lineHa{0.5}{1}{0}
        \lineVa{-1}{-0.5}{0}
    \end{diagram}
}
\newcommand{\TensorAC}[1]{
    \begin{diagram}
        \dobase{0}{0} \tensor{0}{0}{#1}
        \lineHa{-1}{-0.5}{0}
        \lineHa{1}{0.5}{0}
        \lineVa{-1}{-0.5}{0}
    \end{diagram}
}

\newcommand{\tensorL}[3]{
    \draw (-0.5+#1,-0.5+#2) -- (0.25+#1,-0.5+#2) 
    -- (0.5+#1,0+#2) -- (0.25+#1,0.5+#2) 
    -- (-0.5+#1,0.5+#2) -- cycle;
    \draw (#1-0.07,#2) node {#3};
}
\newcommand{\TensorLa}[1]{
    \begin{diagram}
        \dobase{0}{0} \tensorL{0}{0}{#1}
        \lineHa{-1}{-0.5}{0}
        \lineHa{0.5}{1}{0}
        \lineVa{-1}{-0.5}{0}
    \end{diagram}
}

\newcommand{\tensorR}[3]{
    \draw (0.5+#1,-0.5+#2) -- (-0.25+#1,-0.5+#2) 
    -- (-0.5+#1,0+#2) -- (-0.25+#1,0.5+#2) 
    -- (0.5+#1,0.5+#2) -- cycle;
    \draw (#1+0.05,#2) node {#3};
}
\newcommand{\TensorRa}[1]{
    \begin{diagram}
        \dobase{0}{0} \tensorR{0}{0}{#1}
        \lineHa{-1}{-0.5}{0}
        \lineHa{0.5}{1}{0}
        \lineVa{-1}{-0.5}{0}
    \end{diagram}
}

\newcommand{\tensorU}[3]{
    \draw (-0.5+#1,0.5+#2) -- (0.5+#1,0.5+#2) 
    -- (0.5+#1,-0.25+#2) -- (0+#1,-0.5+#2) 
    -- (-0.5+#1,-0.25+#2) -- cycle;
    \draw (#1,#2+0.07) node {#3};
}
\newcommand{\tensorD}[3]{
    \draw (-0.5+#1,-0.5+#2) -- (0.5+#1,-0.5+#2) 
    -- (0.5+#1,0.25+#2) -- (0+#1,0.5+#2) 
    -- (-0.5+#1,0.25+#2) -- cycle;
    \draw (#1,#2-0.05) node {#3};
}

\newcommand{\fuserL}[5]{
    \draw (#1,#2-#3) -- (#1,#2+#3) 
    -- (#1-#4,#2) -- cycle;
    \node at (#1-0.4*#4,#2) {#5};
}
\newcommand{\fuserR}[5]{
    \draw (#1,#2-#3) -- (#1,#2+#3) 
    -- (#1+#4,#2) -- cycle;
    \node at (#1+0.4*#4,#2) {#5};
}

\NewDocumentCommand{\mat}{O{0.5} m m m}{
    \draw (#2,#3) circle (#1);
    \draw (#2,#3) node {#4};
}
\NewDocumentCommand{\Matrixa}{O{0.5} m}{
    \begin{diagram}
        \dobase{0}{0} \mat[#1]{0}{0}{#2}
        \lineHa{#1}{1}{0} \lineHa{-1}{-#1}{0}
    \end{diagram}
}
\NewDocumentCommand{\MatrixC}{O{0.5} m}{
    \begin{diagram}
        \dobase{0}{0} \mat[#1]{0}{0}{#2}
        \lineHa{1}{#1}{0} \lineHa{-1}{-#1}{0}
    \end{diagram}
}

\NewDocumentCommand{\weight}{O{0.5} m m m}
{\begin{scope}[shift={(#2,#3)}]
    \draw (-#1,0) -- (0,#1) -- (#1,0) -- (0,-#1) -- cycle;
    \node at (0,0) {#4};
\end{scope}}

\NewDocumentCommand{\bloba}{m m m m m}{
    % Get the parameters
    \def\xc{#1}
    \def\yc{#2}
    \def\inangles{#3}
    \def\outangles{#4}
    % Draw the circle at the specified coordinates
    \draw (\xc,\yc) circle (0.5);
    % Draw incoming arrows
    \foreach \angle in \inangles {
        \pgfmathsetmacro{\xstart}{\xc + 1.5*cos(\angle)}
        \pgfmathsetmacro{\ystart}{\yc + 1.5*sin(\angle)}
        \pgfmathsetmacro{\xend}{\xc + 0.5*cos(\angle)}
        \pgfmathsetmacro{\yend}{\yc + 0.5*sin(\angle)}
        \draw[midarrow] (\xstart, \ystart) -- (\xend, \yend);
    }
    % Draw outgoing arrows
    \foreach \angle in \outangles {
        \pgfmathsetmacro{\xstart}{\xc + 0.5*cos(\angle)}
        \pgfmathsetmacro{\ystart}{\yc + 0.5*sin(\angle)}
        \pgfmathsetmacro{\xend}{\xc + 1.5*cos(\angle)}
        \pgfmathsetmacro{\yend}{\yc + 1.5*sin(\angle)}
        \draw[midarrow] (\xstart, \ystart) -- (\xend, \yend);
    }
    \draw (#1,#2) node {#5};
}

\NewDocumentCommand{\blob}{m m m m}{
    % Get the parameters
    \def\xc{#1}
    \def\yc{#2}
    \def\angles{#3}
    % Draw the circle at the specified coordinates
    \draw (\xc,\yc) circle (0.5);
    % Draw incoming arrows
    \foreach \angle in \angles {
        \pgfmathsetmacro{\xstart}{\xc + 1.5*cos(\angle)}
        \pgfmathsetmacro{\ystart}{\yc + 1.5*sin(\angle)}
        \pgfmathsetmacro{\xend}{\xc + 0.5*cos(\angle)}
        \pgfmathsetmacro{\yend}{\yc + 0.5*sin(\angle)}
        \draw (\xstart, \ystart) -- (\xend, \yend);
    }
    \draw (#1,#2) node {#4};
}

\newcommand{\closeLeft}[2]
{\begin{scope}[shift={(#1,0)}]
    \draw (-1,0.5) to[out=90,in=180] (0,1.5);
    \draw (0,-1.5) to[out=180,in=270] (-1,-0.5);
    \mat{-1}{0}{#2}
\end{scope}}
\newcommand{\closeLefta}[2]
{\begin{scope}[shift={(#1,0)}]
    \draw[midarrow] (-1,0.5) to[out=90,in=180] (0,1.5);
    \draw[midarrow] (0,-1.5) to[out=180,in=270] (-1,-0.5);
    \mat{-1}{0}{#2}
\end{scope}}

\newcommand{\closeRight}[2]{
    \draw (#1,+1.5) to[out=0,in=90] (#1+1,0.5);
    \draw (#1+1,-0.5) to[out=270,in=0] (#1,-1.5);
    \mat{#1+1}{0}{#2}
}
\newcommand{\closeRighta}[2]{
    \draw[midarrow] (#1+1,0.5) to[out=90,in=0] (#1,+1.5);
    \draw[midarrow] (#1,-1.5) to[out=0,in=-90] (#1+1,-0.5);
    \mat{#1+1}{0}{#2}
}

\newcommand{\colmat}[3]{
    \tensor{#1}{1.5}{#2}
    \tensor{#1}{-1.5}{#3}
    \lineV{-1}{1}{#1}
}
\newcommand{\colmata}[3]{
    \tensor{#1}{1.5}{#2}
    \tensor{#1}{-1.5}{#3}
    \lineVa{-1}{1}{#1}
}

\newcommand{\colmatL}[3]{
    \tensorL{#1}{1.5}{#2}
    \tensorL{#1}{-1.5}{#3}
    \lineV{-1}{1}{#1}
}
\newcommand{\colmatLa}[3]{
    \tensorL{#1}{1.5}{#2}
    \tensorL{#1}{-1.5}{#3}
    \lineVa{-1}{1}{#1}
}

\newcommand{\colmatR}[3]{
    \tensorR{#1}{1.5}{#2}
    \tensorR{#1}{-1.5}{#3}
    \lineV{-1}{1}{#1}
}
\newcommand{\colmatRa}[3]{
    \tensorR{#1}{1.5}{#2}
    \tensorR{#1}{-1.5}{#3}
    \lineVa{-1}{1}{#1}
}

\newcommand{\applyTransferL}[3]{
    \begin{diagram}
        \dobase{0}{0}
        \colmat{0}{#1}{#3} \closeLeft{-0.5}{#2}
        \lineH{0.5}{1}{1.5} \lineH{0.5}{1}{-1.5}
    \end{diagram}
}
\newcommand{\applyTransferLa}[3]{
    \begin{diagram}
        \dobase{0}{0}
        \colmata{0}{#1}{#3} 
        \closeLefta{-0.5}{#2}
        \lineHa{0.5}{1}{1.5} 
        \lineHa{1}{0.5}{-1.5}
    \end{diagram}
}

\newcommand{\applyTransferR}[3]{
    \begin{diagram}
        \dobase{0}{0}
        \colmat{0}{#1}{#3} \closeRight{0.5}{#2}
        \lineH{-0.5}{-1}{1.5} \lineH{-0.5}{-1}{-1.5}
    \end{diagram}
}

\newcommand{\lrtimesMPS}[3]{
    \begin{diagram}
        \dobase{0}{0}
        \lineH{-3}{-2.5}{0}
        \mat{-2}{0}{#1}
        \lineH{-1.5}{-0.5}{0}
        \tensor{0}{0}{#2}
        \lineV{-1}{-0.5}{0}
        \lineH{0.5}{1.5}{0}
        \mat{2}{0}{#3}
        \lineH{2.5}{3}{0}
    \end{diagram}
}
\newcommand{\lrtimesMPSa}[3]{
    \begin{diagram}
        \dobase{0}{0}
        \lineHa{-3}{-2.5}{0}
        \mat{-2}{0}{#1}
        \lineHa{-1.5}{-0.5}{0}
        \tensor{0}{0}{#2}
        \lineVa{-1}{-0.5}{0}
        \lineHa{0.5}{1.5}{0}
        \mat{2}{0}{#3}
        \lineHa{2.5}{3}{0}
    \end{diagram}
}

\newcommand{\ltimesMPS}[2]{
    \begin{diagram}
        \dobase{0}{0}
        \mat{0}{0}{#1}
        \tensor{2}{0}{#2}
        \lineH{-1}{-0.5}{0}
        \lineH{0.5}{1.5}{0}
        \lineH{2.5}{3}{0}
        \lineV{-1}{-0.5}{2}
    \end{diagram}
}
\newcommand{\ltimesMPSa}[2]{
    \begin{diagram}
        \dobase{0}{0}
        \mat{0}{0}{#1}
        \tensor{2}{0}{#2}
        \lineHa{-1}{-0.5}{0}
        \lineHa{0.5}{1.5}{0}
        \lineHa{2.5}{3}{0}
        \lineVa{-1}{-0.5}{2}
    \end{diagram}
}

\newcommand{\rtimesMPS}[2]{
    \begin{diagram}
        \dobase{0}{0}
        \tensor{0}{0}{#1}
        \mat{2}{0}{#2}
        \lineH{-1}{-0.5}{0}
        \lineH{0.5}{1.5}{0}
        \lineH{2.5}{3}{0}
        \lineV{-1}{-0.5}{0}
    \end{diagram}
}
\newcommand{\rtimesMPSa}[2]{
    \begin{diagram}
        \dobase{0}{0}
        \tensor{0}{0}{#1}
        \mat{2}{0}{#2}
        \lineHa{-1}{-0.5}{0}
        \lineHa{0.5}{1.5}{0}
        \lineHa{2.5}{3}{0}
        \lineVa{-1}{-0.5}{0}
    \end{diagram}
}

% --------------------------------
% Special math fonts and notations
% --------------------------------

\newcommand{\graphics}[2]{\includegraphics[totalheight={#1}]{#2}}
\newcommand{\tr}{\operatorname{tr}}
\newcommand{\Abs}[1]{
    \left\lVert #1 \right\rVert
}
\newcommand{\bs}[1]{\boldsymbol{#1}}
\newcommand{\p}{\mathsf{P}}
\newcommand{\h}[1]{\ensuremath{\hat{#1}}}
\renewcommand{\t}[1]{\ensuremath{\tilde{#1}}}
\newcommand{\Z}{\ensuremath{\mathbb{Z}}}
\newcommand{\C}{\ensuremath{\mathbb{C}}}
\newcommand{\R}{\ensuremath{\mathbb{R}}}
\newcommand{\voperator}[1]{\ensuremath{\breve{#1}}}
\newcommand{\manifold}[1]{\ensuremath{\mathcal{#1}}}
\newcommand{\group}[1]{\ensuremath{\mathsf{#1}}}
\newcommand{\vectorspace}[1]{\ensuremath{\mathbb{#1}}}
\newcommand{\algebra}[1]{\ensuremath{\mathfrak{#1}}}
\newcommand{\pf}{\operatorname{pf}}

\newcommand{\Imag}{\mathrm{Im}}
\newcommand{\Real}{\mathrm{Re}}
\newcommand{\sgn}{\mathrm{sgn}}
\newcommand{\T}{\mathsf{T}}
\newcommand{\diag}{\mathrm{diag}}

\newcommand{\e}{\ensuremath{\mathrm{e}}}
\renewcommand{\d}{\ensuremath{\mathrm{d}}}
\renewcommand{\dag}{^\dagger}
\newcommand{\one}{\mathds{1}}
\newcommand{\ol}[1]{\overline{#1}}

\newcommand{\bA}{\bar{A}}
\newcommand{\tA}{\tilde{A}}
\newcommand{\bB}{\bar{B}}
\newcommand{\bG}{\bar{G}}
\newcommand{\tC}{\tilde{C}}

\newcommand{\vect}[1]{\ensuremath{\boldsymbol{#1}}}
\newcommand{\lv}{\ensuremath{\vect{v}_L^\dagger}}
\newcommand{\rv}{\ensuremath{\vect{v}_R}}
\newcommand{\spst}{\ensuremath{\ket{\{s\}}}}
\newcommand{\E}[2]{\ensuremath{E^{#1}_{#2}}}
\renewcommand{\O}[2]{\ensuremath{O^{#1}_{#2}}}
\renewcommand{\H}[4]{\ensuremath{H^{#1#2}_{#3#4}}}
\newcommand{\I}[6]{\ensuremath{I^{#1#2#3}_{#4#5#6}}}
\newcommand{\HH}[6]{\ensuremath{H^{#1#2#3}_{#4#5#6}}}
\newcommand{\J}[4]{\ensuremath{J^{#1#2}_{#3#4}}}
\newcommand{\alphap}{\ensuremath{{\alpha'}}}
\newcommand{\betap}{\ensuremath{{\beta'}}}
\newcommand{\gammap}{\ensuremath{{\gamma'}}}

% group theory
\newcommand{\U}{\ensuremath{\mathrm{U}}}
\newcommand{\SU}{\ensuremath{\mathrm{SU}}}
\newcommand{\su}{\ensuremath{\mathfrak{su}}}
\newcommand{\SO}{\ensuremath{\mathrm{SO}}}
\newcommand{\so}{\ensuremath{\mathfrak{so}}}

% category theory
\newcommand{\Hom}{\ensuremath{\mathrm{Hom}}}

% \renewcommand{\cite}[1]{}
\graphicspath{{./images/}}

\begin{document}

\section{Imaginary time evolution of PEPS}

\begingroup
\def\cirrad{0.1}
\def\recrad{0.1}
\def\wtsize{0.08}
\def\shift{0.3}
\newcommand{\wt}[2]{
    \begin{scope}[shift={(#1,#2)}]
        \draw[fill=yellow] (0,\wtsize) -- (\wtsize,0) 
        -- (0,-\wtsize) -- (-\wtsize,0) -- cycle;
    \end{scope}
}
\newcommand{\pmat}[2]{
    \draw[fill=white] (#1,#2) circle (\wtsize);
}
\newcommand{\drawTbase}[5]{
    \draw[#5] (#1,#2) to (#1+#4,#2);
    \draw[#5] (#1-#4,#2) to (#1,#2);
    \draw[#5] (#1,#2) to (#1,#2+#4);
    \draw[#5] (#1,#2-#4) to (#1,#2);
    \fill[#3] (#1,#2) circle (\cirrad);
}
\newcommand{\drawTket}[3]{
    \draw[midarrow] (#1+\shift,#2+\shift) -- (#1,#2);
    \drawTbase{#1}{#2}{#3}{0.5}{midarrow}
}
\newcommand{\drawTbra}[3]{
    \drawTbase{#1}{#2}{#3}{0.5}{midarrowrev}
    \draw[midarrow] (#1,#2) -- (#1-\shift,#2-\shift);
}
\newcommand{\drawXket}[3]{
    \drawTbase{#1}{#2}{#3}{0.5}{midarrow}
}
\newcommand{\drawXbra}[3]{
    \drawTbase{#1}{#2}{#3}{0.5}{midarrowrev}
}
\newcommand{\drawaR}[3]{
    \draw[midarrow] (#1-0.5,#2) to (#1,#2);
    \draw[midarrow] (#1,#2) to (#1+0.5,#2);
    \draw[midarrow] (#1+\shift,#2+\shift) to (#1,#2);
    \draw[#3, fill=white] (#1,#2) circle (\cirrad);
}
\newcommand{\gatebond}[2]{
    \fill[rounded corners=3, color=orange]
    (#1-\recrad, #2-\recrad) rectangle 
    (#1+1+\recrad, #2+\recrad);
}
\newcommand{\drawgate}{
    \draw[rounded corners, fill=orange, xslant=1] 
    (-0.1,0.3) rectangle ++(1.2,0.4);
    \draw (1,0.5) node {$g_{ij}$};
    \foreach \x in {0,1} 
    {\draw[midarrow] (\x+1,1) to (\x+0.7,+0.7);}
}
\newcommand{\drawlatbase}[1]{
    \def\mydrawT{#1}
    \foreach \x in {0,1,2,3} {
    \foreach \y in {0,1,2,3} {
        \pgfmathparse{\x + \y}
        \let\sumxy\pgfmathresult
        \ifodd\sumxy
            \mydrawT{\x}{\y}{green}
        \else
            \mydrawT{\x}{\y}{blue}
        \fi
        \ifnum\x<3 \wt{\x+0.5}{\y} \fi 
        \ifnum\y<3 \wt{\x}{\y+0.5} \fi
    }}
    \foreach \x in {0,1,2,3} {
    \foreach \y in {0,1,2,3} {
        \ifnum\x<3 \wt{\x+0.5}{\y} \fi 
        \ifnum\y<3 \wt{\x}{\y+0.5} \fi
    }}
}
\newcommand{\drawlatket}{
    \drawlatbase{\drawTket}
}
\newcommand{\drawlatbra}{
    \drawlatbase{\drawTbra}
}

A common way to obtain the ground state of a Hamiltonian is \emph{imaginary time evolution}: 
starting from a random state $\ket{\psi_0}$, the ground state is $\ket{\psi} \approx e^{-\beta H} \ket{\psi_0}$ when $\beta \to \infty$. 
Suppose that the Hamiltonian is the sum of nearest neighbor terms:
\begin{equation}
    H = \sum_{\braket{ij}} H_{ij}. 
\end{equation}
Suppose that the initial state is a (fermionic) PEPS on the bipartite square lattice, which contains two sub-lattices labelled as $A, B$ and four types of bonds (non-equivalent under lattice translation) labelled from 1 to 4:
\begin{equation}
    \ket{\psi_{AB}} = \begin{diagram}[0.8][1.2]
        \dobase{0}{1.5} \drawlatket
        \node[anchor=north west] at (1,1) {$T_A$};
        \node[anchor=north west] at (2,1) {$T_B$};
        \node[anchor=south] at (1.5,1) {$\lambda_1$};
        \node[anchor=east] at (1,1.5) {$\lambda_2$};
        \node[anchor=north] at (0.5,1) {$\lambda_3$};
        \node[anchor=east] at (1,0.5) {$\lambda_4$};
    \end{diagram}. 
\end{equation}
Here we assumed translation invariance of both the Hamiltonian and the PEPS. For convenience of the time evolution process, we place a diagonal Schmidt weight matrix $\lambda_a$ (shown as yellow diamonds) on each type-$a$ bond. 
The infinitesimal evolution operator $e^{-\epsilon H}$ ($\epsilon \ll 1$) can be Trotter decomposed to
\begin{equation}
    e^{-\epsilon H}
    = e^{-\epsilon H_1} \cdots e^{-\epsilon H_4}
    + O(\epsilon^2), 
    \quad
    H_a = \sum_{\braket{ij} \in a} H_{ij}. 
\end{equation}
Note that $H_{ij}$ on different type-$a$ bonds do not overlap. Therefore, for each $H_a$, 
\begin{equation}
    e^{-\epsilon H_a}
    = \prod_{\braket{ij} \in a} g_{ij},
    \quad
    g_{ij} = e^{-\epsilon H_{ij}}
\end{equation}
Each $g_{ij}$ is called a two-body \emph{gate} acting on the neighboring sites $i,j$. Its action on $\ket{\psi(A, B)}$ is
\begin{equation}
    \ket{\psi_G}
    = g_{ij} \ket{\psi_{AB}}
    = \begin{diagram}[0.8][1.2]
        \dobase{0}{1.5} \drawlatket
        \gatebond{1}{1} 
        \node[anchor=north] at (1.5,1) {$G$};
    \end{diagram}, 
\end{equation}
where the 8-leg tensor $G$ is 
\begingroup
\newcommand{\virbonds}{
    \draw[midarrow] (-0.5,0) to (0,0);
    \draw[midarrow] (1,0) to (1.5,0);
    \foreach \x in {0,1} {
        \lineVa{-0.5}{0}{\x}
        \lineVa{0}{0.5}{\x}
        \draw[midarrow] (\x+\shift,\shift) to (\x,0);
    }
    \draw[midarrow] (0,0) to (1,0);
}
\begin{equation}
    \begin{diagram}[0.9][1.4]
        \dobase{0}{0} 
        \virbonds
        \fill[blue] (0,0) circle (\cirrad);
        \fill[green] (1,0) circle (\cirrad);
        \gatebond{0}{0} 
        \node[anchor=north] at (0.5,0) {$G$};
    \end{diagram} = \begin{diagram}[0.9][1.4]
        \dobase{0}{0}
        \drawTket{0}{0}{blue}
        \drawTket{1}{0}{green}
        \wt{0.5}{0}
        \node[anchor=north east] at (0,0) {$T_A$};
        \node[anchor=north] at (0.5,0) {$\lambda_1$};
        \node[anchor=north west] at (1,0) {$T_B$};
        \drawgate
    \end{diagram}
\end{equation}
\endgroup
In $\ket{\psi_G}$, the tensors on sites $i,j$ are replaced by $A', B'$ (while other sites remain unchanged), and the dimension of the bond $\braket{ij}$ is increased to $D' \le d^2 D$ ($d, D$ are the physical and the virtual bond dimensions, respectively), which will grow exponentially as the evolution continues. 
Therefore, we should approximate $\ket{\psi_G}$ with a PEPS $\ket{\psi_{\tilde{A} \tilde{B}}}$ that has the same virtual bond dimension $D$ on the bond $\braket{ij}$. 
The updated tensors $\tilde{T}_A, \tilde{T}_B$ and the weight $\tilde{\lambda}_1$ are determined by minimizing the cost function 
\begin{equation}
\begin{aligned}
    f(\tilde{T}_A, \tilde{T}_B, \tilde{\lambda}_1)
    &= \lVert
        \ket{\psi_{\tilde{A} \tilde{B}}}
        - \ket{\psi_G}
    \rVert^2
    = \braket{
        \psi_{\tilde{A} \tilde{B}} |
        \psi_{\tilde{A} \tilde{B}}
    } - \left(
        \braket{
            \psi_{\tilde{A} \tilde{B}} |
            \psi_G
        } + h.c.
    \right) + \mathrm{const.}
    \\
    &= \begin{diagram}[0.8][1.2]
        \dobase{0}{1.5} 
        \begin{scope}[opacity=0.3, shift={(\shift,\shift)}]
            \drawlatbra
            \node[anchor=south east] 
            at (1,1) {$\tilde{T}^\dagger_A$};
            \node[anchor=south] 
            at (1.5,1) {$\tilde{\lambda}_1$};
            \node[anchor=south west] 
            at (2,1) {$\tilde{T}^\dagger_B$};
        \end{scope}
        \begin{scope}
            \drawlatket
            \node[anchor=north east] 
            at (1,1) {$\tilde{T}_A$};
            \node[anchor=north] 
            at (1.5,1) {$\tilde{\lambda}_1$};
            \node[anchor=north west] 
            at (2,1) {$\tilde{T}_B$};
        \end{scope} 
    \end{diagram} - \left(
        \begin{diagram}[0.8][1.2]
            \dobase{0}{1.5} 
            \begin{scope}[opacity=0.3, shift={(\shift,\shift)}]
                \drawlatbra
                \node[anchor=south east] 
                at (1,1) {$\tilde{T}^\dagger_A$};
                \node[anchor=south] 
                at (1.5,1) {$\tilde{\lambda}_1$};
                \node[anchor=south west] 
                at (2,1) {$\tilde{T}^\dagger_B$};
            \end{scope}
            \begin{scope}
                \drawlatket \gatebond{1}{1} 
                \node[anchor=north] at (1.5,1) {$G$};
            \end{scope}
        \end{diagram} + h.c.
    \right) + \mathrm{const.}
\end{aligned}
\label{eq:full-update-cost}
\end{equation}
The "const" term is $\braket{\psi_G|\psi_G}$, which independent of $\tilde{A}, \tilde{B}$. 
After the tensors $\tilde{A}, \tilde{B}$ are found, we shall replace \emph{all} $A, B$ tensors in the PEPS by $\tilde{A}, \tilde{B}$ to approximate (to the same order as the Trotter decomposition) the effect of the entire $e^{-\epsilon H_1}$. 
The same procedure is also applied to other types of bonds. 

\subsection{Simple update algorithm}

In general, exact evaluation of the cost function is computationally very hard. In the \textit{simple update} algorithm \cite{Jiang2008}, it is assumed that the "\emph{environment tensor}" surrounding the updated bond is simply the direct product of identity and $P$ tensors (draw as small circles) \cite{Gu2013}:
\newcommand{\rhoedgesA}[1]{
    \draw[#1] (0,1) -- ++(0.5,0);
    \draw[#1] (1,0) -- ++(0,0.5);
    \draw[#1] (1,1.5) -- ++(0,0.5);
}
\newcommand{\rhoedgesB}[1]{
    \draw[#1] (2.5,1) -- ++(0.5,0);
    \draw[#1] (2,0) -- ++(0,0.5);
    \draw[#1] (2,1.5) -- ++(0,0.5);
}
\def\rhoshift{0.6}
\newcommand{\rhowts}{
    \foreach \y in {0,2,3} {
    \foreach \x in {0,1,2} {
        \wt{\x+0.5}{\y}
    }}
    \foreach \x in {0,1,2,3} {
    \foreach \y in {0,1,2} {
        \ifnum\x=1
            \ifnum\y=0\else \ifnum\y=1\else
                \wt{\x}{\y+0.5}
            \fi \fi
        \else \ifnum\x=2
            \ifnum\y=0\else \ifnum\y=1\else
                \wt{\x}{\y+0.5}
            \fi \fi
            \else
                \wt{\x}{\y+0.5}
            \fi
        \fi
    }}
}
\newcommand{\drawrho}[1]{
    \def\mydrawT{#1}
    \foreach \x in {0,1,2,3} {
    \foreach \y in {0,1,2,3} {
        \pgfmathparse{\x + \y}
        \let\sumxy\pgfmathresult
        \ifnum\y=1
            \ifnum\x=1 \else \ifnum\x=2 \else
                \ifodd\sumxy
                    \mydrawT{\x}{\y}{green}
                \else
                    \mydrawT{\x}{\y}{blue}
                \fi
            \fi \fi 
        \else
            \ifodd\sumxy
                \mydrawT{\x}{\y}{green}
            \else
                \mydrawT{\x}{\y}{blue}
            \fi
        \fi
    }} \rhowts
}
\newcommand{\suenvA}{
    \begin{scope}
        \rhoedgesA{midarrow}
    \end{scope}
    \foreach \x/\y in {0/1, 1/0}{
        \draw (\x,\y) -- ++(\rhoshift, \rhoshift);
    }
    \foreach \x/\y in {1/2}{
        \draw (\x,\y) -- ++(\rhoshift, \rhoshift);
        \pmat{\x+\rhoshift/2}{\y+\rhoshift/2}
    }
    \begin{scope}[opacity=0.3,shift={(\rhoshift,\rhoshift)}]
        \rhoedgesA{midarrowrev}
    \end{scope}
}
\newcommand{\suenvB}{
    \begin{scope}
        \rhoedgesB{midarrow}
    \end{scope}
    \foreach \x/\y in {2/0}{
        \draw (\x,\y) -- ++(\rhoshift, \rhoshift);
    }
    \foreach \x/\y in {2/2, 3/1}{
        \draw (\x,\y) -- ++(\rhoshift, \rhoshift);
        \pmat{\x+\rhoshift/2}{\y+\rhoshift/2}
    }
    \begin{scope}[opacity=0.3,shift={(\rhoshift,\rhoshift)}]
        \rhoedgesB{midarrowrev}
    \end{scope}
}
\newcommand{\suenv}{
    \suenvA \suenvB
}
\begin{equation}
    \begin{diagram}[0.8][1.2]
        \dobase{0}{1.2}
        \begin{scope}[opacity=0.3, shift={(\shift,\shift)}]
            \drawrho{\drawTbra}
        \end{scope}
        \begin{scope}
            \drawrho{\drawTket}
        \end{scope} 
    \end{diagram}
    = \begin{diagram}[0.8][1.2]
        \dobase{1}{1+\rhoshift/2} \suenv
    \end{diagram}
\end{equation}
This drastically simplifies the cost function Eq. \eqref{eq:full-update-cost} to
\begingroup
\newcommand{\drawABket}{
    \drawTket{1}{1}{blue}
    \drawTket{2}{1}{green}
}
\newcommand{\drawABbra}{
    \drawTbra{1}{1}{blue}
    \drawTbra{2}{1}{green}
}
\newcommand{\envwts}{
    \wt{0.5}{1} \wt{1.5}{1} \wt{2.5}{1}
    \wt{1}{0.5} \wt{1}{1.5}
    \wt{2}{0.5} \wt{2}{1.5}
}
\begin{equation}
    f(\tilde{T}_A, \tilde{T}_B, \tilde{\lambda}_1)
    \approx \begin{diagram}[0.8][1]
        \dobase{1}{1+\rhoshift/2} \suenv
        \begin{scope}[opacity=0.3,shift={(\rhoshift,\rhoshift)}]
            \drawABbra \envwts
            \node[anchor=south east] 
            at (1,1) {$\tilde{T}^\dagger_A$};
            \node[anchor=south] 
            at (1.5,1) {$\tilde{\lambda}_1$};
            \node[anchor=south west] 
            at (2,1) {$\tilde{T}^\dagger_B$};
        \end{scope} 
        \begin{scope}
            \drawABket \envwts
            \node[anchor=north east] 
            at (1,1) {$\tilde{T}_A$};
            \node[anchor=north] 
            at (1.5,1) {$\tilde{\lambda}_1$};
            \node[anchor=north west] 
            at (2,1) {$\tilde{T}_B$};
        \end{scope} 
    \end{diagram} - \left(
        \begin{diagram}[0.8][1]
            \dobase{1}{1+\rhoshift/2} \suenv
            \begin{scope}[opacity=0.3,shift={(\rhoshift,\rhoshift)}]
                \drawABbra \envwts 
                \node[anchor=south east] 
                at (1,1) {$\tilde{T}^\dagger_A$};
                \node[anchor=south] 
                at (1.5,1) {$\tilde{\lambda}_1$};
                \node[anchor=south west] 
                at (2,1) {$\tilde{T}^\dagger_B$};
            \end{scope}
            \begin{scope}
                \drawABket \envwts \gatebond{1}{1}
                \node[anchor=south] at (1.5,1) {$G$};
            \end{scope} 
        \end{diagram} + h.c.
    \right) + \mathrm{const.}
    \label{eq:simple-update-cost}
\end{equation}
\endgroup
Then minimizing $f$ is equivalent to the low-rank approximation problem of the tensor $G$ with the surrounding weights $\lambda_2, \lambda_3, \lambda_4$ absorbed. It can be solved by SVD in the following steps: 
\begin{enumerate}
    \item The weights $\lambda_2, \lambda_3, \lambda_4$ surrounding the type-1 bond to be updated are absorbed into the old tensors $T_A$ and $T_B$:
    \begin{equation}
        \def\surround{
            \draw[midarrow] (-1,0) -- (-0.5,0);
            \draw[midarrow] (0.5,0) -- (1,0);
            \draw[midarrow] (0,0.5) -- (0,1);
            \draw[midarrow] (0,-1) -- (0,-0.5);
        }
        \begin{diagram}[0.8][1.5]
            \dobase{0}{0}
            \drawTket{0}{0}{blue}
            \node[anchor=north west] at (0,0) {$T_A$};
            \surround
            \wt{-0.5}{0} \wt{0}{-0.5} \wt{0}{0.5}
        \end{diagram} = \begin{diagram}[0.8][1.5]
            \dobase{0}{0}
            \drawTket{0}{0}{blue}
            \node[anchor=north east] at (0,0) {$A$};
        \end{diagram}, \quad
        \begin{diagram}[0.8][1.5]
            \dobase{0}{0}
            \drawTket{0}{0}{green}
            \node[anchor=north west] at (0,0) {$T_B$};
            \surround
            \wt{0.5}{0} \wt{0}{-0.5} \wt{0}{0.5}
        \end{diagram} = \begin{diagram}[0.8][1.5]
            \dobase{0}{0}
            \drawTket{0}{0}{green}
            \node[anchor=north east] at (0,0) {$B$};
        \end{diagram}.
        \label{eq:absorb-wt}
    \end{equation}
    
    \item To reduce SVD computational cost, we first apply QR and LQ decomposition to $\tilde{A}, \tilde{B}$ respectively as
    \begin{equation}
        \begin{diagram}[0.8][1.5]
            \dobase{0}{0}
            \drawTket{0}{0}{blue}
            \node[anchor=north east] at (0,0) {$A$};
        \end{diagram} = \begin{diagram}[0.8][1.5]
            \dobase{0}{0}
            \drawXket{0}{0}{blue} \drawaR{0.5}{0}{blue};
            \node[anchor=north east] at (0,0) {$X$};
            \node[anchor=north] at (0.5,0) {$a_R$};
        \end{diagram}, 
        \quad
        \begin{diagram}[0.8][1.5]
            \dobase{0}{0}
            \drawTket{0}{0}{green}
            \node[anchor=north east] at (0,0) {$B$};
        \end{diagram} = \begin{diagram}[0.8][1.5]
            \dobase{0}{0}
            \drawXket{0}{0}{green} \drawaR{-0.5}{0}{green};
            \node[anchor=north east] at (0,0) {$Y$};
            \node[anchor=north] at (-0.5,0) {$b_L$};
        \end{diagram}.
    \end{equation}
    The tensors $X, Y$ are unitary in the sense that
    \begin{equation}
        \begin{diagram}[0.8][1.2]
            \dobase{0}{1+\rhoshift/2} 
            \begin{scope}[opacity=0.3, shift={(\rhoshift,\rhoshift)}]
                \drawXbra{1}{1}{blue}
                \node[anchor=south west] at (1,1) {$X^\dagger$};
            \end{scope}
            \drawXket{1}{1}{blue}
            \node[anchor=north east] at (1,1) {$X$};
            \suenvA
        \end{diagram} = \begin{diagram}[0.8][1.2]
            \dobase{0}{\rhoshift/2}
            \draw[midarrow] (0,0) -- (0.5,0);
            \draw (0,0) -- ++(\rhoshift, \rhoshift);
            \begin{scope}[opacity=0.3, shift={(\rhoshift,\rhoshift)}]
                \draw[midarrowrev] (0,0) -- (0.5,0);
            \end{scope}
        \end{diagram}
        \quad , \qquad
        \begin{diagram}[0.8][1.2]
            \dobase{2}{1+\rhoshift/2}
            \begin{scope}[opacity=0.3, shift={(\rhoshift,\rhoshift)}]
                \drawXbra{2}{1}{green}
                \node[anchor=south west] at (2,1) {$Y^\dagger$};
            \end{scope}
            \drawXket{2}{1}{green} 
            \node[anchor=north east] at (2,1) {$Y$};
            \suenvB
        \end{diagram} = \begin{diagram}[0.8][1.2]
            \dobase{0}{\rhoshift/2}
            \draw[midarrow] (0,0) -- (0.5,0);
            \draw (0.5,0) -- ++(\rhoshift, \rhoshift);
            \begin{scope}[opacity=0.3, shift={(\rhoshift,\rhoshift)}]
                \draw[midarrowrev] (0,0) -- (0.5,0);
            \end{scope}
            \pmat{0.5+\rhoshift/2}{\rhoshift/2}
        \end{diagram}
    \end{equation}
    The gate $g_{ij}$ now acts on the 3-leg tensors $a_R$ and $b_L$ as \cite{Phien2015}
    \begin{equation}
        \begin{diagram}[0.8][1.5]
            \dobase{0}{0}
            \drawTket{0}{0}{blue}
            \drawTket{1}{0}{green}
            \wt{0.5}{0}
            \node[anchor=north east] at (0,0) {$T_A$};
            \node[anchor=north] at (0.5,0) {$\lambda_1$};
            \node[anchor=north west] at (1,0) {$T_B$};
            \drawgate
        \end{diagram} = \begin{diagram}[0.8][1.5]
            \dobase{0}{0}
            \drawXket{-0.5}{0}{blue} \drawaR{0}{0}{blue}
            \drawaR{1}{0}{green} \drawXket{1.5}{0}{green}
            \wt{0.5}{0}
            \node[anchor=north east] at (-0.5,0) {$X$};
            \node[anchor=north] at (0,0) {$a_R$};
            \node[anchor=north] at (0.5,0) {$\lambda_1$};
            \node[anchor=north] at (1,0) {$b_L$};
            \node[anchor=north west] at (1.5,0) {$Y$};
            \drawgate
        \end{diagram}. 
    \end{equation}
    Then we perform SVD, obtaining the updated weight $\tilde{\lambda}_1$ and 3-leg tensors $\tilde{a}_R$, and $\tilde{b}_L$ as
    \begin{equation}
        \begin{diagram}[0.8][1.5]
            \dobase{0}{0}
            \drawaR{0}{0}{blue} \drawaR{1}{0}{green}
            \wt{0.5}{0}
            \node[anchor=north] at (0,0) {$a_R$};
            \node[anchor=north] at (0.5,0) {$\lambda_1$};
            \node[anchor=north] at (1,0) {$b_L$};
            \drawgate
            \draw[dashed, red] (0.2,-0.3) -- ++(1.5,1.5);
        \end{diagram}
        = \begin{diagram}[0.8][1.5]
            \dobase{0}{0}
            \drawaR{0}{0}{blue} \drawaR{1}{0}{green}
            \wt{0.5}{0}
            \node[anchor=north] at (0,0) {$\tilde{a}_R$};
            \node[anchor=north] at (0.5,0) {$\tilde{\lambda}_1$};
            \node[anchor=north] at (1,0) {$\tilde{b}_L$};
        \end{diagram}. 
    \end{equation}
    
    \item To control the virtual bond dimension, we truncate the new weight $\tilde{\lambda}_1$ by keeping only the largest $D$ singular values. 
    Here, $D$ can be different from the virtual bond dimension $D_0$ of the initial state $\ket{\psi_0}$. 
    Note that the singular values of both the even and odd sectors are sorted together. 
    The even (or odd) dimension $D_e$ (or $D_o$) of the virtual index ($D_e+D_o = D$) is the number of these $D$ singular values that come from the even (or odd) sector of $\Lambda$. 
    
    \item The new $\tilde{\lambda}_1$ is normalized so that the maximum singular value is $1$. 
    The new $\tilde{a}_R, \tilde{b}_L$ tensors are absorbed back into $X, Y$ to produce the new $\tilde{A}, \tilde{B}$ tensors
    \begin{equation}
        \begin{diagram}[0.8][1.5]
            \dobase{0}{0}
            \drawXket{-0.5}{0}{blue} \drawaR{0}{0}{blue}
            \drawaR{1}{0}{green} \drawXket{1.5}{0}{green}
            \wt{0.5}{0}
            \node[anchor=north east] at (-0.5,0) {$X$};
            \node[anchor=north] at (0,0) {$\tilde{a}_R$};
            \node[anchor=north] at (0.5,0) {$\tilde{\lambda}_1$};
            \node[anchor=north] at (1,0) {$\tilde{b}_L$};
            \node[anchor=north west] at (1.5,0) {$Y$};
        \end{diagram} = \begin{diagram}[0.8][1.5]
            \dobase{0}{0}
            \drawTket{0}{0}{blue}
            \drawTket{1}{0}{green}
            \wt{0.5}{0}
            \node[anchor=north east] at (0,0) {$\tilde{A}$};
            \node[anchor=north] at (0.5,0) {$\tilde{\lambda}_1$};
            \node[anchor=north west] at (1,0) {$\tilde{B}$};
        \end{diagram}.
    \end{equation}
    
    \item The absorbed environment weights $\lambda_2, \lambda_3, \lambda_4$ are then restored by reversing Eq. \eqref{eq:absorb-wt} to obtain the updated tensors $\tilde{T}_A, \tilde{T}_B$. 

    \item Steps $1$ to $5$ are repeated for each of the four types of bonds. 
\end{enumerate}
\endgroup

The update stops when the change of the averaged weight is sufficiently small, or more precisely,
\begin{equation}
    \delta \Lambda(n)
    \equiv \frac{1}{D} \sum_{i=1}^D |
        \lambda^{(n)}_i
        - \lambda^{(n-1)}_i
    | \lesssim 10^{-10},
\end{equation}
where $\lambda^{(n)}_i$ is the $i$-th averaged weight after the $n$-th round of simple update.

\end{document}
