\documentclass[11pt]{article}
\usepackage[a4paper, margin=2cm]{geometry}
\usepackage{amsmath}
\usepackage{amssymb}
\usepackage{amsthm}
\usepackage{dsfont}
\usepackage{bbm}
\usepackage{multirow}
\usepackage{array}
\usepackage{diagbox}
\usepackage{makecell}
\newcolumntype{C}[1]{>{\centering\arraybackslash}m{#1}}
% \usepackage{charter}
\usepackage{fontspec}
% \usepackage{unicode-math}
\setmainfont{Charter}
% \setmathfont{XCharter-Math.otf}
% \usepackage{newtxtext,newtxmath}
\usepackage{braket}
\usepackage{slashed}
\usepackage{color}
\usepackage[table]{xcolor}
\usepackage[mathscr]{euscript}
\usepackage{graphicx}
\usepackage{placeins}
\usepackage{floatrow}
\floatsetup[table]{capposition=top}
\usepackage[caption=false]{subfig}
\usepackage[export]{adjustbox}
\floatsetup[figure]{style=plain,subcapbesideposition=top}
\usepackage{xfrac}
\usepackage{bm}
\usepackage{microtype}
\usepackage{commath}
\usepackage{mathtools}
\usepackage{enumitem}
\usepackage{xparse}
\usepackage[
    colorlinks=true, urlcolor=blue,
    linkcolor=blue, citecolor=blue
]{hyperref}% add hypertext capabilities
\usepackage{tikz}
\usepackage{ifthen}
\usepackage[framemethod=TikZ]{mdframed}
\usepackage[version=4]{mhchem}
\usepackage[numbers]{natbib}
\usepackage[nottoc]{tocbibind}
\usepackage[english]{babel}
\usepackage[autostyle, english = american]{csquotes}
\MakeOuterQuote{"}
\usepackage{setspace}

\usetikzlibrary{decorations.markings}
\usetikzlibrary{calc, positioning, arrows.meta}
\tikzset{midarrow/.style={
    decoration={markings, mark=at position 0.55 with {\arrow{>}}},
    postaction={decorate}
}}
\tikzset{midarrowrev/.style={
    decoration={markings, mark=at position 0.45 with {\arrow{<}}},
    postaction={decorate}
}}

% \allowdisplaybreaks

\theoremstyle{remark}
\newtheorem*{remark}{Remark}

\NewDocumentEnvironment{diagram}{O{0.68} O{0.75}}{
    \begin{tikzpicture}[
        baseline = (X.base),
        every node/.style={scale=#1}, scale=#2
    ]
}{\end{tikzpicture}}

\definecolor{green}{RGB}{50, 180, 50}
\definecolor{blue}{RGB}{20, 30, 250}

\newcommand{\pos}[2]{\begin{matrix}
    #1 \\ #2
\end{matrix}}
\newcommand{\dobase}[2]{
    \draw (#1,#2) node (X) {$\phantom{X}$};
}

\NewDocumentCommand{\lineH}{O{black} m m m}{
    \draw[color=#1] (#2,#4) -- (#3,#4);
}
\NewDocumentCommand{\lineHa}{O{black} m m m}{
    \draw[midarrow, color=#1] (#2,#4) -- (#3,#4);
}

\NewDocumentCommand{\lineV}{O{black} m m m}{
    \draw[color=#1] (#4,#2) -- (#4,#3);
}
\NewDocumentCommand{\lineVa}{O{black} m m m}{
    \draw[midarrow, color=#1] (#4,#2) -- (#4,#3);
}

\newcommand{\contrL}[3]{
    \draw (#3,#1) to[out=180,in=180] (#3,#2);
}
\newcommand{\contrLa}[3]{
    \draw[midarrow] (#3,#1) to[out=180,in=180] (#3,#2);
}

\newcommand{\contrR}[3]{
    \draw (#3,#1) to[out=0,in=0] (#3,#2);
}
\newcommand{\contrRa}[3]{
    \draw[midarrow] (#3,#1) to[out=0,in=0] (#3,#2);
}

\NewDocumentCommand{\rect}{m m m m m o}{
    \IfNoValueTF{#6}{
        \draw[rounded corners] 
        (#1 - #3/2, #2 - #4/2) 
        rectangle (#1 + #3/2, #2 + #4/2);
    }{
        \draw[rounded corners, fill=#6] 
        (#1 - #3/2, #2 - #4/2) 
        rectangle (#1 + #3/2, #2 + #4/2);
    }
    \node[align=center] at (#1,#2) {#5};
}

\newcommand{\tensor}[3]{
    \rect{#1}{#2}{1}{1}{#3}
}
\newcommand{\Tensora}[1]{
    \begin{diagram}
        \dobase{0}{0} \tensor{0}{0}{#1}
        \lineHa{-1}{-0.5}{0}
        \lineHa{0.5}{1}{0}
        \lineVa{-1}{-0.5}{0}
    \end{diagram}
}
\newcommand{\TensorAC}[1]{
    \begin{diagram}
        \dobase{0}{0} \tensor{0}{0}{#1}
        \lineHa{-1}{-0.5}{0}
        \lineHa{1}{0.5}{0}
        \lineVa{-1}{-0.5}{0}
    \end{diagram}
}

\newcommand{\tensorL}[3]{
    \draw (-0.5+#1,-0.5+#2) -- (0.25+#1,-0.5+#2) 
    -- (0.5+#1,0+#2) -- (0.25+#1,0.5+#2) 
    -- (-0.5+#1,0.5+#2) -- cycle;
    \draw (#1-0.07,#2) node {#3};
}
\newcommand{\TensorLa}[1]{
    \begin{diagram}
        \dobase{0}{0} \tensorL{0}{0}{#1}
        \lineHa{-1}{-0.5}{0}
        \lineHa{0.5}{1}{0}
        \lineVa{-1}{-0.5}{0}
    \end{diagram}
}

\newcommand{\tensorR}[3]{
    \draw (0.5+#1,-0.5+#2) -- (-0.25+#1,-0.5+#2) 
    -- (-0.5+#1,0+#2) -- (-0.25+#1,0.5+#2) 
    -- (0.5+#1,0.5+#2) -- cycle;
    \draw (#1+0.05,#2) node {#3};
}
\newcommand{\TensorRa}[1]{
    \begin{diagram}
        \dobase{0}{0} \tensorR{0}{0}{#1}
        \lineHa{-1}{-0.5}{0}
        \lineHa{0.5}{1}{0}
        \lineVa{-1}{-0.5}{0}
    \end{diagram}
}

\newcommand{\tensorU}[3]{
    \draw (-0.5+#1,0.5+#2) -- (0.5+#1,0.5+#2) 
    -- (0.5+#1,-0.25+#2) -- (0+#1,-0.5+#2) 
    -- (-0.5+#1,-0.25+#2) -- cycle;
    \draw (#1,#2+0.07) node {#3};
}
\newcommand{\tensorD}[3]{
    \draw (-0.5+#1,-0.5+#2) -- (0.5+#1,-0.5+#2) 
    -- (0.5+#1,0.25+#2) -- (0+#1,0.5+#2) 
    -- (-0.5+#1,0.25+#2) -- cycle;
    \draw (#1,#2-0.05) node {#3};
}

\newcommand{\fuserL}[5]{
    \draw (#1,#2-#3) -- (#1,#2+#3) 
    -- (#1-#4,#2) -- cycle;
    \node at (#1-0.4*#4,#2) {#5};
}
\newcommand{\fuserR}[5]{
    \draw (#1,#2-#3) -- (#1,#2+#3) 
    -- (#1+#4,#2) -- cycle;
    \node at (#1+0.4*#4,#2) {#5};
}

\NewDocumentCommand{\mat}{O{0.5} m m m}{
    \draw (#2,#3) circle (#1);
    \draw (#2,#3) node {#4};
}
\NewDocumentCommand{\Matrixa}{O{0.5} m}{
    \begin{diagram}
        \dobase{0}{0} \mat[#1]{0}{0}{#2}
        \lineHa{#1}{1}{0} \lineHa{-1}{-#1}{0}
    \end{diagram}
}
\NewDocumentCommand{\MatrixC}{O{0.5} m}{
    \begin{diagram}
        \dobase{0}{0} \mat[#1]{0}{0}{#2}
        \lineHa{1}{#1}{0} \lineHa{-1}{-#1}{0}
    \end{diagram}
}

\NewDocumentCommand{\weight}{O{0.5} m m m}
{\begin{scope}[shift={(#2,#3)}]
    \draw (-#1,0) -- (0,#1) -- (#1,0) -- (0,-#1) -- cycle;
    \node at (0,0) {#4};
\end{scope}}

\NewDocumentCommand{\bloba}{m m m m m}{
    % Get the parameters
    \def\xc{#1}
    \def\yc{#2}
    \def\inangles{#3}
    \def\outangles{#4}
    % Draw the circle at the specified coordinates
    \draw (\xc,\yc) circle (0.5);
    % Draw incoming arrows
    \foreach \angle in \inangles {
        \pgfmathsetmacro{\xstart}{\xc + 1.5*cos(\angle)}
        \pgfmathsetmacro{\ystart}{\yc + 1.5*sin(\angle)}
        \pgfmathsetmacro{\xend}{\xc + 0.5*cos(\angle)}
        \pgfmathsetmacro{\yend}{\yc + 0.5*sin(\angle)}
        \draw[midarrow] (\xstart, \ystart) -- (\xend, \yend);
    }
    % Draw outgoing arrows
    \foreach \angle in \outangles {
        \pgfmathsetmacro{\xstart}{\xc + 0.5*cos(\angle)}
        \pgfmathsetmacro{\ystart}{\yc + 0.5*sin(\angle)}
        \pgfmathsetmacro{\xend}{\xc + 1.5*cos(\angle)}
        \pgfmathsetmacro{\yend}{\yc + 1.5*sin(\angle)}
        \draw[midarrow] (\xstart, \ystart) -- (\xend, \yend);
    }
    \draw (#1,#2) node {#5};
}

\NewDocumentCommand{\blob}{m m m m}{
    % Get the parameters
    \def\xc{#1}
    \def\yc{#2}
    \def\angles{#3}
    % Draw the circle at the specified coordinates
    \draw (\xc,\yc) circle (0.5);
    % Draw incoming arrows
    \foreach \angle in \angles {
        \pgfmathsetmacro{\xstart}{\xc + 1.5*cos(\angle)}
        \pgfmathsetmacro{\ystart}{\yc + 1.5*sin(\angle)}
        \pgfmathsetmacro{\xend}{\xc + 0.5*cos(\angle)}
        \pgfmathsetmacro{\yend}{\yc + 0.5*sin(\angle)}
        \draw (\xstart, \ystart) -- (\xend, \yend);
    }
    \draw (#1,#2) node {#4};
}

\newcommand{\closeLeft}[2]
{\begin{scope}[shift={(#1,0)}]
    \draw (-1,0.5) to[out=90,in=180] (0,1.5);
    \draw (0,-1.5) to[out=180,in=270] (-1,-0.5);
    \mat{-1}{0}{#2}
\end{scope}}
\newcommand{\closeLefta}[2]
{\begin{scope}[shift={(#1,0)}]
    \draw[midarrow] (-1,0.5) to[out=90,in=180] (0,1.5);
    \draw[midarrow] (0,-1.5) to[out=180,in=270] (-1,-0.5);
    \mat{-1}{0}{#2}
\end{scope}}

\newcommand{\closeRight}[2]{
    \draw (#1,+1.5) to[out=0,in=90] (#1+1,0.5);
    \draw (#1+1,-0.5) to[out=270,in=0] (#1,-1.5);
    \mat{#1+1}{0}{#2}
}
\newcommand{\closeRighta}[2]{
    \draw[midarrow] (#1+1,0.5) to[out=90,in=0] (#1,+1.5);
    \draw[midarrow] (#1,-1.5) to[out=0,in=-90] (#1+1,-0.5);
    \mat{#1+1}{0}{#2}
}

\newcommand{\colmat}[3]{
    \tensor{#1}{1.5}{#2}
    \tensor{#1}{-1.5}{#3}
    \lineV{-1}{1}{#1}
}
\newcommand{\colmata}[3]{
    \tensor{#1}{1.5}{#2}
    \tensor{#1}{-1.5}{#3}
    \lineVa{-1}{1}{#1}
}

\newcommand{\colmatL}[3]{
    \tensorL{#1}{1.5}{#2}
    \tensorL{#1}{-1.5}{#3}
    \lineV{-1}{1}{#1}
}
\newcommand{\colmatLa}[3]{
    \tensorL{#1}{1.5}{#2}
    \tensorL{#1}{-1.5}{#3}
    \lineVa{-1}{1}{#1}
}

\newcommand{\colmatR}[3]{
    \tensorR{#1}{1.5}{#2}
    \tensorR{#1}{-1.5}{#3}
    \lineV{-1}{1}{#1}
}
\newcommand{\colmatRa}[3]{
    \tensorR{#1}{1.5}{#2}
    \tensorR{#1}{-1.5}{#3}
    \lineVa{-1}{1}{#1}
}

\newcommand{\applyTransferL}[3]{
    \begin{diagram}
        \dobase{0}{0}
        \colmat{0}{#1}{#3} \closeLeft{-0.5}{#2}
        \lineH{0.5}{1}{1.5} \lineH{0.5}{1}{-1.5}
    \end{diagram}
}
\newcommand{\applyTransferLa}[3]{
    \begin{diagram}
        \dobase{0}{0}
        \colmata{0}{#1}{#3} 
        \closeLefta{-0.5}{#2}
        \lineHa{0.5}{1}{1.5} 
        \lineHa{1}{0.5}{-1.5}
    \end{diagram}
}

\newcommand{\applyTransferR}[3]{
    \begin{diagram}
        \dobase{0}{0}
        \colmat{0}{#1}{#3} \closeRight{0.5}{#2}
        \lineH{-0.5}{-1}{1.5} \lineH{-0.5}{-1}{-1.5}
    \end{diagram}
}

\newcommand{\lrtimesMPS}[3]{
    \begin{diagram}
        \dobase{0}{0}
        \lineH{-3}{-2.5}{0}
        \mat{-2}{0}{#1}
        \lineH{-1.5}{-0.5}{0}
        \tensor{0}{0}{#2}
        \lineV{-1}{-0.5}{0}
        \lineH{0.5}{1.5}{0}
        \mat{2}{0}{#3}
        \lineH{2.5}{3}{0}
    \end{diagram}
}
\newcommand{\lrtimesMPSa}[3]{
    \begin{diagram}
        \dobase{0}{0}
        \lineHa{-3}{-2.5}{0}
        \mat{-2}{0}{#1}
        \lineHa{-1.5}{-0.5}{0}
        \tensor{0}{0}{#2}
        \lineVa{-1}{-0.5}{0}
        \lineHa{0.5}{1.5}{0}
        \mat{2}{0}{#3}
        \lineHa{2.5}{3}{0}
    \end{diagram}
}

\newcommand{\ltimesMPS}[2]{
    \begin{diagram}
        \dobase{0}{0}
        \mat{0}{0}{#1}
        \tensor{2}{0}{#2}
        \lineH{-1}{-0.5}{0}
        \lineH{0.5}{1.5}{0}
        \lineH{2.5}{3}{0}
        \lineV{-1}{-0.5}{2}
    \end{diagram}
}
\newcommand{\ltimesMPSa}[2]{
    \begin{diagram}
        \dobase{0}{0}
        \mat{0}{0}{#1}
        \tensor{2}{0}{#2}
        \lineHa{-1}{-0.5}{0}
        \lineHa{0.5}{1.5}{0}
        \lineHa{2.5}{3}{0}
        \lineVa{-1}{-0.5}{2}
    \end{diagram}
}

\newcommand{\rtimesMPS}[2]{
    \begin{diagram}
        \dobase{0}{0}
        \tensor{0}{0}{#1}
        \mat{2}{0}{#2}
        \lineH{-1}{-0.5}{0}
        \lineH{0.5}{1.5}{0}
        \lineH{2.5}{3}{0}
        \lineV{-1}{-0.5}{0}
    \end{diagram}
}
\newcommand{\rtimesMPSa}[2]{
    \begin{diagram}
        \dobase{0}{0}
        \tensor{0}{0}{#1}
        \mat{2}{0}{#2}
        \lineHa{-1}{-0.5}{0}
        \lineHa{0.5}{1.5}{0}
        \lineHa{2.5}{3}{0}
        \lineVa{-1}{-0.5}{0}
    \end{diagram}
}

% --------------------------------
% Special math fonts and notations
% --------------------------------

\newcommand{\graphics}[2]{\includegraphics[totalheight={#1}]{#2}}
\newcommand{\tr}{\operatorname{tr}}
\newcommand{\Abs}[1]{
    \left\lVert #1 \right\rVert
}
\newcommand{\bs}[1]{\boldsymbol{#1}}
\newcommand{\p}{\mathsf{P}}
\newcommand{\h}[1]{\ensuremath{\hat{#1}}}
\renewcommand{\t}[1]{\ensuremath{\tilde{#1}}}
\newcommand{\Z}{\ensuremath{\mathbb{Z}}}
\newcommand{\C}{\ensuremath{\mathbb{C}}}
\newcommand{\R}{\ensuremath{\mathbb{R}}}
\newcommand{\voperator}[1]{\ensuremath{\breve{#1}}}
\newcommand{\manifold}[1]{\ensuremath{\mathcal{#1}}}
\newcommand{\group}[1]{\ensuremath{\mathsf{#1}}}
\newcommand{\vectorspace}[1]{\ensuremath{\mathbb{#1}}}
\newcommand{\algebra}[1]{\ensuremath{\mathfrak{#1}}}
\newcommand{\pf}{\operatorname{pf}}

\newcommand{\Imag}{\mathrm{Im}}
\newcommand{\Real}{\mathrm{Re}}
\newcommand{\sgn}{\mathrm{sgn}}
\newcommand{\T}{\mathsf{T}}
\newcommand{\diag}{\mathrm{diag}}

\newcommand{\e}{\ensuremath{\mathrm{e}}}
\renewcommand{\d}{\ensuremath{\mathrm{d}}}
\renewcommand{\dag}{^\dagger}
\newcommand{\one}{\mathds{1}}
\newcommand{\ol}[1]{\overline{#1}}

\newcommand{\bA}{\bar{A}}
\newcommand{\tA}{\tilde{A}}
\newcommand{\bB}{\bar{B}}
\newcommand{\bG}{\bar{G}}
\newcommand{\tC}{\tilde{C}}

\newcommand{\vect}[1]{\ensuremath{\boldsymbol{#1}}}
\newcommand{\lv}{\ensuremath{\vect{v}_L^\dagger}}
\newcommand{\rv}{\ensuremath{\vect{v}_R}}
\newcommand{\spst}{\ensuremath{\ket{\{s\}}}}
\newcommand{\E}[2]{\ensuremath{E^{#1}_{#2}}}
\renewcommand{\O}[2]{\ensuremath{O^{#1}_{#2}}}
\renewcommand{\H}[4]{\ensuremath{H^{#1#2}_{#3#4}}}
\newcommand{\I}[6]{\ensuremath{I^{#1#2#3}_{#4#5#6}}}
\newcommand{\HH}[6]{\ensuremath{H^{#1#2#3}_{#4#5#6}}}
\newcommand{\J}[4]{\ensuremath{J^{#1#2}_{#3#4}}}
\newcommand{\alphap}{\ensuremath{{\alpha'}}}
\newcommand{\betap}{\ensuremath{{\beta'}}}
\newcommand{\gammap}{\ensuremath{{\gamma'}}}

% group theory
\newcommand{\U}{\ensuremath{\mathrm{U}}}
\newcommand{\SU}{\ensuremath{\mathrm{SU}}}
\newcommand{\su}{\ensuremath{\mathfrak{su}}}
\newcommand{\SO}{\ensuremath{\mathrm{SO}}}
\newcommand{\so}{\ensuremath{\mathfrak{so}}}

% category theory
\newcommand{\Hom}{\ensuremath{\mathrm{Hom}}}

\graphicspath{{./images/}}

\def\pad{0.1}
\newcommand{\tensora}[2]{
    \draw[midarrow] (#1,#2) -- +(1,0);
    \draw[midarrow] (#1-1,#2) -- +(1,0);
    \draw[midarrow] (#1,#2) -- +(0,1);
    \draw[midarrow] (#1,#2-1) -- +(0,1);
    \draw[fill=green] (#1,#2) circle (\pad);
}
\NewDocumentCommand{\ctmnw}{m m O{$C^1$}}{
    \draw[midarrow] (#1,#2) -- +(1,0);
    \draw[midarrow] (#1,#2-1) -- +(0,1);
    \node[anchor=south] at (#1,#2+0.1) {#3};
    \draw[fill=orange] (#1-\pad,#2-\pad) rectangle (#1+\pad,#2+\pad);
}
\NewDocumentCommand{\ctmne}{m m O{$C^2$}}{
    \draw[midarrow] (#1-1,#2) -- +(1,0);
    \draw[midarrow] (#1,#2-1) -- +(0,1);
    \node[anchor=south] at (#1,#2+0.1) {#3};
    \draw[fill=orange] (#1-\pad,#2-\pad) rectangle (#1+\pad,#2+\pad);
}
\NewDocumentCommand{\ctmsw}{m m O{$C^4$}}{
    \draw[midarrow] (#1,#2) -- +(1,0);
    \draw[midarrow] (#1,#2) -- +(0,1);
    \node[anchor=north] at (#1,#2-0.1) {#3};
    \draw[fill=orange] (#1-\pad,#2-\pad) rectangle (#1+\pad,#2+\pad);
}
\NewDocumentCommand{\ctmse}{m m O{$C^3$}}{
    \draw[midarrow] (#1-1,#2) -- +(1,0);
    \draw[midarrow] (#1,#2) -- +(0,1);
    \node[anchor=north] at (#1,#2-0.1) {#3};
    \draw[fill=orange] (#1-\pad,#2-\pad) rectangle (#1+\pad,#2+\pad);
}

\NewDocumentCommand{\ctmn}{m m O{$T^1$}}{
    \draw[midarrow] (#1-1,#2) -- +(1,0);
    \draw[midarrow] (#1,#2) -- +(1,0);
    \draw[midarrow] (#1,#2-1) -- +(0,1);
    \node[anchor=south] at (#1,#2+0.1) {#3};
    \draw[fill=yellow] (#1,#2) circle (\pad);
}
\NewDocumentCommand{\ctme}{m m O{$T^2$}}{
    \draw[midarrow] (#1,#2-1) -- +(0,1);
    \draw[midarrow] (#1,#2) -- +(0,1);
    \draw[midarrow] (#1-1,#2) -- +(1,0);
    \node[anchor=west] at (#1+0.1,#2) {#3};
    \draw[fill=yellow] (#1,#2) circle (\pad);
}
\NewDocumentCommand{\ctms}{m m O{$T^3$}}{
    \draw[midarrow] (#1-1,#2) -- +(1,0);
    \draw[midarrow] (#1,#2) -- +(1,0);
    \draw[midarrow] (#1,#2) -- +(0,1);
    \node[anchor=north] at (#1,#2-0.1) {#3};
    \draw[fill=yellow] (#1,#2) circle (\pad);
}
\NewDocumentCommand{\ctmw}{m m O{$T^4$}}{
    \draw[midarrow] (#1,#2-1) -- +(0,1);
    \draw[midarrow] (#1,#2) -- +(0,1);
    \draw[midarrow] (#1,#2) -- +(1,0);
    \node[anchor=east] at (#1-0.1,#2) {#3};
    \draw[fill=yellow] (#1,#2) circle (\pad);
}

\NewDocumentCommand{\projl}{m m O{$P^a$}}{
    \draw[midarrow] (#1-1,#2+0.5) -- ++(1,0);
    \draw[midarrow] (#1-1,#2-0.5) -- ++(1,0);
    \draw[midarrow] (#1+0.3,#2) -- ++(0.7,0);
    \draw[fill=cyan] (#1,#2-0.7) -- (#1+0.3,#2) -- (#1,#2+0.7) -- cycle;
    \node[anchor=south west] at (#1+0.2,#2) {#3};
}
\NewDocumentCommand{\projr}{m m O{$P^b$}}{
    \draw[midarrow] (#1,#2+0.5) -- ++(1,0);
    \draw[midarrow] (#1,#2-0.5) -- ++(1,0);
    \draw[midarrow] (#1-1,#2) -- ++(0.7,0);
    \draw[fill=cyan] (#1,#2-0.7) -- (#1-0.3,#2) -- (#1,#2+0.7) -- cycle;
    \node[anchor=south east] at (#1-0.2,#2) {#3};
}

\begin{document}

\section{CTMRG algorithm}

Consider an iPEPS with $N_x \times N_y$ unit cell.
\begin{equation}
    \ket{\Psi} = \begin{diagram}[0.8][1.0]
        \dobase{0}{0} 
        \foreach \x in {-1,...,1} \foreach \y in {-1,...,1}
        {\tensora{\x}{\y} \draw[midarrow] (\x+0.3,\y+0.3) -- (\x,\y);}
        \foreach \x/\xlabel in {-1/x-1,0/x,1/x+1}
        {\node at (\x,-2.2) {$\xlabel$};}
        \foreach \y/\ylabel in {-1/y-1,0/y,1/y+1}
        {\node at (2.5,\y) {$\ylabel$};}
        \node[anchor=north east] at (0,0) {$A_{x,y}$};
    \end{diagram}, \qquad
    A_{x+N_x,y} = A_{x,y+N_y} = A_{x,y}
\end{equation}
We choose the arrows on $A_{x,y}$ to point in the direction of increasing $x$, $y$. To calculate the norm of $\ket{\Psi}$ or measure the expectation value in $\ket{\Psi}$, we need to contract the physical axis of $A_{x,y}$ with $A^\dagger_{x,y}$, leading to the reduced tensor $M_{x,y}$:
\begin{equation}
    \def\shift{0.15}
    \begin{diagram}[1.0][1.0]
        \dobase{0}{0} \tensora{0}{0}
        \node[anchor=south east] at (0,0) {$M_{x,y}$};
    \end{diagram}
    = \begin{diagram}[1.0][1.0]
        \dobase{0}{0} 
        \draw[midarrow] (\shift,\shift) -- (-\shift,-\shift);
        \tensora{-\shift}{-\shift}
        \node[anchor=north east] at (-\shift,-\shift) {$A_{x,y}$};
        \begin{scope}[opacity=0.5]
            \tensora{\shift}{\shift}
            \node[anchor=south west] at (\shift,\shift) {$A'^\dagger_{x,y}$};
        \end{scope}
    \end{diagram}
\end{equation}
The arrows on virtual legs of $A^\dagger_{x,y}$ are flipped to be the same as $A_{x,y}$. In the CTMRG algorithm to contract the network $\ket{\Psi|\Psi}$, the environment of each tensor $a_{x,y}$ ($x=1,...,N_x$, $y=1,...,N_y$) is represented by the corner transfer matrices (CTMs)
\begin{equation}
    \begin{diagram}[1.0][0.7]
        \dobase{0}{0}
        \foreach \angle in {0,90,180,270}{\begin{scope}[rotate=\angle]
            \draw[fill=yellow!40, rounded corners] (-0.3,0.7) rectangle ++(0.6,2.5);
            \draw[fill=orange!40, rounded corners] (0.7,0.7) rectangle ++(2.5,2.5);
        \end{scope}}
        \foreach \x in {-2,...,2} \foreach \y in {-2,...,2}
        {\tensora{\x}{\y}}
        \node at (0,-3.7) {$x$};
        \node at (-3.7,0) {$y$};
    \end{diagram}
    \quad \to \quad \begin{diagram}[0.9][1.3]
        \dobase{0}{0} \tensora{0}{0}
        \node[anchor=south west] at (0,0) {$M$};
        \ctmn{0}{1} \ctme{1}{0} \ctms{0}{-1} \ctmw{-1}{0}
        \ctmnw{-1}{1} \ctmne{1}{1} \ctmse{1}{-1} \ctmsw{-1}{-1}
        \foreach \x/\xlabel in {-1/x-1,0/x,1/x+1}
        {\node at (\x,-1.7) {$\xlabel$};}
        \foreach \y/\ylabel in {-1/y-1,0/y,1/y+1}
        {\node at (2.0,\y) {$\ylabel$};}
    \end{diagram}
\end{equation}

\subsection{The RG algorithm}

The \emph{down move} updates the down edge: for each $x = 1, ..., N_x$ and $y = 1, ..., N_y$, the renormalization step for the CTMs $C^4, T^3, C^3$ is \cite{Corboz2016}
\begin{equation}
    \begin{diagram}[0.9][1.0]
        \dobase{0}{0} 
        \foreach \x/\xlabel in {-1/x-1,0/x,1/x+1}
        {\node at (\x,1.3) {$\xlabel$};}
        \foreach \y/\ylabel in {-1/y-1,0/y}
        {\node at (-2.2,\y) {$\ylabel$};}
        \node[anchor=south west] at (0,0) {$M$};
        \ctmw{-1}{0} \tensora{0}{0} \ctme{1}{0}
        \ctmsw{-1}{-1} \ctms{0}{-1} \ctmse{1}{-1}
    \end{diagram} 
    \quad \to \quad \begin{diagram}[1.0][1.0]
        \dobase{0}{0} 
        \foreach \x/\xlabel in {-1/x-1,0/x,1/x+1}
        {\node at (\x,1.3) {$\xlabel$};}
        \node at (2.2,0) {$y$};
        \ctmsw{-1}{0} \ctms{0}{0} \ctmse{1}{0}
    \end{diagram}
\end{equation}
The two rows are compressed together using \emph{projectors} $P^a$ and $P^b$:
\begingroup
\def\scale{0.9}
\def\tscale{1.0}
\begin{equation}
\begin{gathered}
    \begin{diagram}[\tscale][\scale]
        \dobase{0}{0} \ctmsw{0}{0}[$C^4_{x-1,y}$]
    \end{diagram} = \begin{diagram}[\tscale][\scale]
        \dobase{0}{-0.5} 
        \ctmsw{0}{-1}[$C^4_{x-1,y-1}$] 
        \ctmw{0}{0}[$T^4_{x-1,y}$] 
        \projl{1}{-0.5}[$P^a_{x-1,y-1}$]
    \end{diagram}, 
    \qquad \qquad
    \begin{diagram}[\tscale][\scale]
        \dobase{0}{0} \ctmse{0}{0}[$C^3_{x+1,y}$]
    \end{diagram} = \begin{diagram}[\tscale][\scale]
        \dobase{0}{-0.5} 
        \projr{-1}{-0.5}[$P^b_{x,y-1}$]
        \ctmse{0}{-1}[$C^3_{x+1,y-1}$] 
        \ctme{0}{0}[$T^2_{x+1,y}$]
    \end{diagram}
    \\
    \begin{diagram}[\tscale][\scale]
        \dobase{0}{0} \ctms{0}{0}[$T^3_{x,y}$]
    \end{diagram} = \begin{diagram}[\tscale][\scale]
        \dobase{0}{-0.5} 
        \ctms{0}{-1}[$T^3_{x,y-1}$]
        \tensora{0}{0}
        \node[anchor=south west] at (0,0) {$M_{x,y}$};
        \projr{-1}{-0.5}[$P^b_{x-1,y-1}$]
        \projl{1}{-0.5}[$P^a_{x,y-1}$]
    \end{diagram}
\end{gathered}
\end{equation}
\endgroup
where $P^a_{x,y}, P^b_{x,y}$ satisfy the \emph{approximate} identity
\begin{equation}
    \begin{diagram}[1.0][0.9]
        \dobase{0}{0}
        \projr{-0.5}{0}[$P^b_{x,y}$]
        \projl{0.5}{0}[$P^a_{x,y}$]
    \end{diagram}
    \quad \approx \quad 
    \begin{diagram}[1.0][0.9]
        \dobase{0}{0}
        \draw[midarrow] (-0.5,0) -- (0.5,0);
    \end{diagram}
    \label{eq:proj-iden}
\end{equation}

\subsubsection*{Construction of Projectors}

The performance of the CTMRG algorithm depends heavily on the choice of projectors.
In Ref. \cite{Corboz2014}, the projectors $\{P^a_{x,y}\}, \{P^b_{x,y}\}$ are constructed in the following way: 
\begin{itemize}
    \item Construct a $2 \times 2$ unit cell surrounded a loop of the CTMs, with $M_{x,y}$ in the lower-left corner. 
    \begin{equation}
        \def\nx{1} \def\ny{1}
        \begin{diagram}[0.9][1.2]
            \dobase{0}{0}
            \foreach \x in {0,...,\nx} \foreach \y in {0,...,\ny}
            {\tensora{\x}{\y}}
            \foreach \x in {0,...,\nx} {\ctms{\x}{-1} \ctmn{\x}{\ny+1}}
            \foreach \y in {0,...,\ny} {\ctmw{-1}{\y} \ctme{\nx+1}{\y}}
            \ctmnw{-1}{\ny+1} \ctmne{\nx+1}{\ny+1}
            \ctmsw{-1}{-1} \ctmse{\nx+1}{-1}
            \draw[red,dashed] (0.5,-1.3) -- (0.5,0.3);
            \node[red,anchor=north] at (0.5,-1.3) {cut};
            \draw[red,dashed] (0.5,0.7) -- (0.5,2.3);
            \node[red,anchor=south] at (0.5,2.3) {cut};

            \foreach \x/\xlabel in {-1/x-1,0/x,1/x+1,2/x+2}
            {\node at (\x,-1.8) {$\xlabel$};}
            \foreach \y/\ylabel in {-1/y-1,0/y,1/y+1,2/y+2}
            {\node at (-2.2,\y) {$\ylabel$};}
        \end{diagram}
    \end{equation}

    \item To obtain $P^{a,b}_{x,y-1}$, the network is cut through between the $x$-th and the $(x+1)$-th column. Then we perform QR and LQ decomposition for the left half $C_L$ and the right half $C_R$ to obtain the tensors $R_{x,y-1}$, $L_{x,y-1}$:
    \begin{equation}
        \begin{diagram}[0.8][1.3]
            \dobase{0}{0} 
            \rect{0}{0}{0.6}{1.5}{$C_L$}[lightgray]
            \foreach \y in {-0.6,-0.2,0.2,0.6}
            {\draw[midarrow] (0.3,\y) -- (0.7,\y);}
        \end{diagram} = \begin{diagram}[0.8][1.2]
            \dobase{0}{0}
            \foreach \y in {-0.6,-0.4,0.4,0.6}
            {\draw[midarrow] (0.3,\y) -- (0.7,\y);}
            \rect{0}{0.5}{0.6}{0.6}{$Q$}
            \rect{0}{-0.5}{0.6}{0.6}{$R$}
            \draw[midarrow] (-0.3,0.5) to[out=180,in=180] (-0.3,-0.5);
        \end{diagram}, \qquad \begin{diagram}[0.8][1.2]
            \dobase{0}{0}
            \foreach \y in {-0.6,-0.4,0.4,0.6}
            {\draw[midarrow] (-0.7,\y) -- (-0.3,\y);}
            \rect{0}{0.5}{0.6}{0.6}{$\tilde{Q}$}
            \rect{0}{-0.5}{0.6}{0.6}{$L$}
            \draw[midarrow] (0.3,-0.5) to[out=0,in=0] (0.3,0.5);
        \end{diagram} = \begin{diagram}[0.8][1.3]
            \dobase{0}{0} 
            \rect{0}{0}{0.6}{1.5}{$C_R$}[lightgray]
            \foreach \y in {-0.6,-0.2,0.2,0.6}
            {\draw[midarrow] (-0.7,\y) -- (-0.3,\y);}
        \end{diagram}
    \end{equation}

    \item To truncate the bond between $R$ and $L$, we perform SVD on $R L$:
    \begin{equation}
        \begin{diagram}[0.8][1.2]
            \dobase{0}{0}
            \rect{-0.5}{0}{0.6}{0.6}{$R$}
            \rect{0.5}{0}{0.6}{0.6}{$L$}
            \foreach \y in {-0.1,0.1}
            {\draw[midarrow] (-0.2,\y) -- (0.2,\y);}
            \draw[midarrow] (-1.1,0) -- (-0.8,0);
            \draw[midarrow] (0.8,0) -- (1.1,0);
        \end{diagram} \approx \begin{diagram}[0.8][1.2]
            \rect{-0.7}{0}{0.6}{0.6}{$U$}
            \rect{0.7}{0}{0.6}{0.6}{$V^\dagger$}
            \draw[midarrow] (-1.5,0) -- (-1,0);
            \draw[midarrow] (-0.4,0) -- (-0.1,0);
            \draw[midarrow] (0.1,0) -- (0.4,0);
            \draw[midarrow] (1,0) -- (1.5,0);
            \weight[0.1]{0}{0}{}
            \node[anchor=south] at (0,0.1) {$s$};
        \end{diagram}
    \end{equation}
    The singular value spectrum $s$ is truncated to wanted dimension. This also implies
    \begin{equation}
        \begin{diagram}[0.8][1.2]
            \dobase{0}{0}
            \rect{-0.5}{0}{0.6}{0.6}{$L^{-1}$}
            \rect{0.5}{0}{0.6}{0.6}{$R^{-1}$}
            \foreach \y in {-0.1,0.1}
            {\draw[midarrow] (-0.2,\y) -- (0.2,\y);}
            \draw[midarrow] (-1.1,0) -- (-0.8,0);
            \draw[midarrow] (0.8,0) -- (1.1,0);
        \end{diagram} \approx \begin{diagram}[0.8][1.2]
            \rect{-0.7}{0}{0.6}{0.6}{$V$}
            \rect{0.7}{0}{0.6}{0.6}{$U^\dagger$}
            \draw[midarrow] (-1.5,0) -- (-1,0);
            \draw[midarrow] (-0.4,0) -- (-0.1,0);
            \draw[midarrow] (0.1,0) -- (0.4,0);
            \draw[midarrow] (1,0) -- (1.5,0);
            \weight[0.1]{0}{0}{}
            \node[anchor=south] at (0,0.1) {$1/s$};
        \end{diagram}
    \end{equation}

    \item The projectors $P^a_{x,y-1}, P^b_{x,y-1}$ are then defined as
    \begingroup
    \newcommand{\mywt}[1]{
        \weight[0.1]{#1}{0}{}
        \draw[midarrow] (#1+0.1,0) -- ++(0.3,0);
        \draw[midarrow] (#1-0.4,0) -- ++(0.3,0);
    }
    \begin{equation}
        \begin{diagram}[1.0][0.8]
            \dobase{0}{0} \projl{0}{0}[$P^a_{x,y-1}$]
        \end{diagram} = \begin{diagram}[0.8][1.2]
            \dobase{0}{0} \rect{0}{0}{0.6}{0.6}{$L$}
            \rect{1}{0}{0.6}{0.6}{$V$}
            \foreach \y in {-0.1,0.1}
            {\draw[midarrow] (-0.7,\y) -- (-0.3,\y);}
            \draw[midarrow] (0.3,0) -- (0.7,0);
            \mywt{1.7}
            \node[anchor=south] at (1.7,0.1) {$s^{-1/2}$};
        \end{diagram}, \quad
        \begin{diagram}[1.0][0.8]
            \dobase{0}{0} \projr{0}{0}[$P^b_{x,y-1}$]
        \end{diagram} = \begin{diagram}[0.8][1.2]
            \dobase{0}{0} \rect{0}{0}{0.6}{0.6}{$U^\dagger$}
            \rect{1}{0}{0.6}{0.6}{$R$}
            \foreach \y in {-0.1,0.1}
            {\draw[midarrow] (1.3,\y) -- (1.7,\y);}
            \draw[midarrow] (0.3,0) -- (0.7,0);
            \mywt{-0.7}
            \node[anchor=south] at (-0.7,0.1) {$s^{-1/2}$};
        \end{diagram}
    \end{equation}
    \endgroup
    They satisfy the approximate identity Eq. \eqref{eq:proj-iden}.
\end{itemize}
This procedure is equivalent to the following \cite{Fishman2018,Mortier2024}:
\begin{itemize}
    \item We directly perform SVD on $C_L C_R$:
    \begin{equation}
        \begin{diagram}[0.8][1.3]
            \dobase{0}{0} 
            \rect{0}{0}{0.6}{1.5}{$C_L$}[lightgray]
            \rect{2}{0}{0.6}{1.5}{$C_R$}[lightgray]
            \foreach \y in {0.2,0.6} {
                \draw[midarrow] (0.3,\y) -- (0.7,\y);
                \draw[midarrow] (1.3,\y) -- (1.7,\y);
            }
            \foreach \y in {-0.6,-0.2}
            {\draw[midarrow] (0.3,\y) -- (1.7,\y);}
        \end{diagram}
        \quad \approx \quad \begin{diagram}[0.8][1.3]
            \dobase{0}{0}
            \rect{-1}{0.4}{0.6}{0.6}{$U$}
            \rect{1}{0.4}{0.6}{0.6}{$V^\dagger$}
            \weight[0.1]{0}{-0.4}{}
            \node at (0,-0.7) {$s$};
            \foreach \y in {0.3,0.5} {
                \draw[midarrow] (-0.7,\y) -- (-0.3,\y);
                \draw[midarrow] (0.3,\y) -- (0.7,\y);
            }
            \draw[midarrow] (-1,0.1) to[out=-90,in=180] (-0.1,-0.4);
            \draw[midarrow] (0.1,-0.4) to[out=0,in=-90] (1,0.1);
        \end{diagram}
    \end{equation}

    \item The projectors $P^a_{x,y-1}, P^b_{x,y-1}$ are defined as
    \begin{equation}
        \begin{diagram}[1.0][0.8]
            \dobase{0}{0} \projl{0}{0}[$P^a_{x,y-1}$]
        \end{diagram} = \begin{diagram}[0.8][1.3]
            \dobase{0}{-0.4} 
            \rect{0}{0.45}{0.6}{0.6}{$V$}
            \rect{1.4}{0}{0.6}{1.5}{$C_R$}[lightgray]
            \foreach \y in {0.3, 0.6}{
                \draw[midarrow] (0.3,\y) -- (0.6,\y);
                \draw (0.7,\y) circle (0.1);
                \draw[midarrow] (0.8,\y) -- (1.1,\y);
            }
            \foreach \y in {-0.5,-0.2}
            {\draw[midarrow] (0,\y) -- (1.1,\y);}
            \weight[0.1]{0.7}{1.2}{}
            \draw[midarrow] (0,0.75) 
            to[out=90,in=180] (0.6,1.2);
            \draw[midarrow] (0.8,1.2) -- (2,1.2);
            \draw[midarrow] (2,1.2) -- (2,-0.4);
            \draw[midarrow] (2,-0.4) -- (2.5,-0.4);
        \end{diagram}, 
        \qquad
        \begin{diagram}[1.0][0.8]
            \dobase{0}{0} \projr{0}{0}[$P^b_{x,y-1}$]
        \end{diagram} = \begin{diagram}[0.8][1.3]
            \dobase{0}{-0.4} 
            \rect{1.4}{0.45}{0.6}{0.6}{$U^\dagger$}
            \rect{0}{0}{0.6}{1.5}{$C_L$}[lightgray]
            \foreach \y in {0.3, 0.6}{
                \draw[midarrow] (0.3,\y) -- (0.6,\y);
                \draw (0.7,\y) circle (0.1);
                \draw[midarrow] (0.8,\y) -- (1.1,\y);
            }
            \foreach \y in {-0.5,-0.2}
            {\draw[midarrow] (0.3,\y) -- (1.5,\y);}
            \weight[0.1]{0.7}{1.2}{}
            \draw[midarrow] (0.8,1.2)
            to[out=0,in=90] (1.4,0.75);
            \draw[midarrow] (-1,-0.4) -- (-0.5,-0.4);
            \draw[midarrow] (-0.5,-0.4) -- (-0.5,1.2);
            \draw[midarrow] (-0.5,1.2) -- (0.6,1.2);
        \end{diagram}
    \end{equation}
    The circles are the $P$ tensors needed to cancel unwanted fermion signs. 
\end{itemize}

Similarly, we can perform \emph{up/left/right moves} on the corresponding edges. After we update the four edges, the obtained CTMs should be normalized. 

\subsection{Momentum space calculation}

\cite{Ponsioen2022}

\subsection{Manual calculation of gradients}

\cite{Corboz2016}

\subsection{Auto-differentiation of CTMRG}

\cite{Liao2019,Ponsioen2022,Francuz2023}

\bibliographystyle{ieeetr}
\bibliography{./refs}

\end{document}
