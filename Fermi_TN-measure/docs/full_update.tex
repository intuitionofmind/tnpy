\documentclass[11pt]{article}
\usepackage[a4paper, margin=2cm]{geometry}
\usepackage{amsmath}
\usepackage{amssymb}
\usepackage{amsthm}
\usepackage{dsfont}
\usepackage{bbm}
\usepackage{multirow}
\usepackage{array}
\usepackage{diagbox}
\usepackage{makecell}
\newcolumntype{C}[1]{>{\centering\arraybackslash}m{#1}}
% \usepackage{charter}
\usepackage{fontspec}
% \usepackage{unicode-math}
\setmainfont{Charter}
% \setmathfont{XCharter-Math.otf}
% \usepackage{newtxtext,newtxmath}
\usepackage{braket}
\usepackage{slashed}
\usepackage{color}
\usepackage[table]{xcolor}
\usepackage[mathscr]{euscript}
\usepackage{graphicx}
\usepackage{placeins}
\usepackage{floatrow}
\floatsetup[table]{capposition=top}
\usepackage[caption=false]{subfig}
\usepackage[export]{adjustbox}
\floatsetup[figure]{style=plain,subcapbesideposition=top}
\usepackage{xfrac}
\usepackage{bm}
\usepackage{microtype}
\usepackage{commath}
\usepackage{mathtools}
\usepackage{enumitem}
\usepackage{xparse}
\usepackage[
    colorlinks=true, urlcolor=blue,
    linkcolor=blue, citecolor=blue
]{hyperref}% add hypertext capabilities
\usepackage{tikz}
\usepackage{ifthen}
\usepackage[framemethod=TikZ]{mdframed}
\usepackage[version=4]{mhchem}
\usepackage[numbers]{natbib}
\usepackage[nottoc]{tocbibind}
\usepackage[english]{babel}
\usepackage[autostyle, english = american]{csquotes}
\MakeOuterQuote{"}
\usepackage{setspace}

\usetikzlibrary{decorations.markings}
\usetikzlibrary{calc, positioning, arrows.meta}
\tikzset{midarrow/.style={
    decoration={markings, mark=at position 0.55 with {\arrow{>}}},
    postaction={decorate}
}}
\tikzset{midarrowrev/.style={
    decoration={markings, mark=at position 0.45 with {\arrow{<}}},
    postaction={decorate}
}}

% \allowdisplaybreaks

\theoremstyle{remark}
\newtheorem*{remark}{Remark}

\NewDocumentEnvironment{diagram}{O{0.68} O{0.75}}{
    \begin{tikzpicture}[
        baseline = (X.base),
        every node/.style={scale=#1}, scale=#2
    ]
}{\end{tikzpicture}}

\definecolor{green}{RGB}{50, 180, 50}
\definecolor{blue}{RGB}{20, 30, 250}

\newcommand{\pos}[2]{\begin{matrix}
    #1 \\ #2
\end{matrix}}
\newcommand{\dobase}[2]{
    \draw (#1,#2) node (X) {$\phantom{X}$};
}

\NewDocumentCommand{\lineH}{O{black} m m m}{
    \draw[color=#1] (#2,#4) -- (#3,#4);
}
\NewDocumentCommand{\lineHa}{O{black} m m m}{
    \draw[midarrow, color=#1] (#2,#4) -- (#3,#4);
}

\NewDocumentCommand{\lineV}{O{black} m m m}{
    \draw[color=#1] (#4,#2) -- (#4,#3);
}
\NewDocumentCommand{\lineVa}{O{black} m m m}{
    \draw[midarrow, color=#1] (#4,#2) -- (#4,#3);
}

\newcommand{\contrL}[3]{
    \draw (#3,#1) to[out=180,in=180] (#3,#2);
}
\newcommand{\contrLa}[3]{
    \draw[midarrow] (#3,#1) to[out=180,in=180] (#3,#2);
}

\newcommand{\contrR}[3]{
    \draw (#3,#1) to[out=0,in=0] (#3,#2);
}
\newcommand{\contrRa}[3]{
    \draw[midarrow] (#3,#1) to[out=0,in=0] (#3,#2);
}

\NewDocumentCommand{\rect}{m m m m m o}{
    \IfNoValueTF{#6}{
        \draw[rounded corners] 
        (#1 - #3/2, #2 - #4/2) 
        rectangle (#1 + #3/2, #2 + #4/2);
    }{
        \draw[rounded corners, fill=#6] 
        (#1 - #3/2, #2 - #4/2) 
        rectangle (#1 + #3/2, #2 + #4/2);
    }
    \node[align=center] at (#1,#2) {#5};
}

\newcommand{\tensor}[3]{
    \rect{#1}{#2}{1}{1}{#3}
}
\newcommand{\Tensora}[1]{
    \begin{diagram}
        \dobase{0}{0} \tensor{0}{0}{#1}
        \lineHa{-1}{-0.5}{0}
        \lineHa{0.5}{1}{0}
        \lineVa{-1}{-0.5}{0}
    \end{diagram}
}
\newcommand{\TensorAC}[1]{
    \begin{diagram}
        \dobase{0}{0} \tensor{0}{0}{#1}
        \lineHa{-1}{-0.5}{0}
        \lineHa{1}{0.5}{0}
        \lineVa{-1}{-0.5}{0}
    \end{diagram}
}

\newcommand{\tensorL}[3]{
    \draw (-0.5+#1,-0.5+#2) -- (0.25+#1,-0.5+#2) 
    -- (0.5+#1,0+#2) -- (0.25+#1,0.5+#2) 
    -- (-0.5+#1,0.5+#2) -- cycle;
    \draw (#1-0.07,#2) node {#3};
}
\newcommand{\TensorLa}[1]{
    \begin{diagram}
        \dobase{0}{0} \tensorL{0}{0}{#1}
        \lineHa{-1}{-0.5}{0}
        \lineHa{0.5}{1}{0}
        \lineVa{-1}{-0.5}{0}
    \end{diagram}
}

\newcommand{\tensorR}[3]{
    \draw (0.5+#1,-0.5+#2) -- (-0.25+#1,-0.5+#2) 
    -- (-0.5+#1,0+#2) -- (-0.25+#1,0.5+#2) 
    -- (0.5+#1,0.5+#2) -- cycle;
    \draw (#1+0.05,#2) node {#3};
}
\newcommand{\TensorRa}[1]{
    \begin{diagram}
        \dobase{0}{0} \tensorR{0}{0}{#1}
        \lineHa{-1}{-0.5}{0}
        \lineHa{0.5}{1}{0}
        \lineVa{-1}{-0.5}{0}
    \end{diagram}
}

\newcommand{\tensorU}[3]{
    \draw (-0.5+#1,0.5+#2) -- (0.5+#1,0.5+#2) 
    -- (0.5+#1,-0.25+#2) -- (0+#1,-0.5+#2) 
    -- (-0.5+#1,-0.25+#2) -- cycle;
    \draw (#1,#2+0.07) node {#3};
}
\newcommand{\tensorD}[3]{
    \draw (-0.5+#1,-0.5+#2) -- (0.5+#1,-0.5+#2) 
    -- (0.5+#1,0.25+#2) -- (0+#1,0.5+#2) 
    -- (-0.5+#1,0.25+#2) -- cycle;
    \draw (#1,#2-0.05) node {#3};
}

\newcommand{\fuserL}[5]{
    \draw (#1,#2-#3) -- (#1,#2+#3) 
    -- (#1-#4,#2) -- cycle;
    \node at (#1-0.4*#4,#2) {#5};
}
\newcommand{\fuserR}[5]{
    \draw (#1,#2-#3) -- (#1,#2+#3) 
    -- (#1+#4,#2) -- cycle;
    \node at (#1+0.4*#4,#2) {#5};
}

\NewDocumentCommand{\mat}{O{0.5} m m m}{
    \draw (#2,#3) circle (#1);
    \draw (#2,#3) node {#4};
}
\NewDocumentCommand{\Matrixa}{O{0.5} m}{
    \begin{diagram}
        \dobase{0}{0} \mat[#1]{0}{0}{#2}
        \lineHa{#1}{1}{0} \lineHa{-1}{-#1}{0}
    \end{diagram}
}
\NewDocumentCommand{\MatrixC}{O{0.5} m}{
    \begin{diagram}
        \dobase{0}{0} \mat[#1]{0}{0}{#2}
        \lineHa{1}{#1}{0} \lineHa{-1}{-#1}{0}
    \end{diagram}
}

\NewDocumentCommand{\weight}{O{0.5} m m m}
{\begin{scope}[shift={(#2,#3)}]
    \draw (-#1,0) -- (0,#1) -- (#1,0) -- (0,-#1) -- cycle;
    \node at (0,0) {#4};
\end{scope}}

\NewDocumentCommand{\bloba}{m m m m m}{
    % Get the parameters
    \def\xc{#1}
    \def\yc{#2}
    \def\inangles{#3}
    \def\outangles{#4}
    % Draw the circle at the specified coordinates
    \draw (\xc,\yc) circle (0.5);
    % Draw incoming arrows
    \foreach \angle in \inangles {
        \pgfmathsetmacro{\xstart}{\xc + 1.5*cos(\angle)}
        \pgfmathsetmacro{\ystart}{\yc + 1.5*sin(\angle)}
        \pgfmathsetmacro{\xend}{\xc + 0.5*cos(\angle)}
        \pgfmathsetmacro{\yend}{\yc + 0.5*sin(\angle)}
        \draw[midarrow] (\xstart, \ystart) -- (\xend, \yend);
    }
    % Draw outgoing arrows
    \foreach \angle in \outangles {
        \pgfmathsetmacro{\xstart}{\xc + 0.5*cos(\angle)}
        \pgfmathsetmacro{\ystart}{\yc + 0.5*sin(\angle)}
        \pgfmathsetmacro{\xend}{\xc + 1.5*cos(\angle)}
        \pgfmathsetmacro{\yend}{\yc + 1.5*sin(\angle)}
        \draw[midarrow] (\xstart, \ystart) -- (\xend, \yend);
    }
    \draw (#1,#2) node {#5};
}

\NewDocumentCommand{\blob}{m m m m}{
    % Get the parameters
    \def\xc{#1}
    \def\yc{#2}
    \def\angles{#3}
    % Draw the circle at the specified coordinates
    \draw (\xc,\yc) circle (0.5);
    % Draw incoming arrows
    \foreach \angle in \angles {
        \pgfmathsetmacro{\xstart}{\xc + 1.5*cos(\angle)}
        \pgfmathsetmacro{\ystart}{\yc + 1.5*sin(\angle)}
        \pgfmathsetmacro{\xend}{\xc + 0.5*cos(\angle)}
        \pgfmathsetmacro{\yend}{\yc + 0.5*sin(\angle)}
        \draw (\xstart, \ystart) -- (\xend, \yend);
    }
    \draw (#1,#2) node {#4};
}

\newcommand{\closeLeft}[2]
{\begin{scope}[shift={(#1,0)}]
    \draw (-1,0.5) to[out=90,in=180] (0,1.5);
    \draw (0,-1.5) to[out=180,in=270] (-1,-0.5);
    \mat{-1}{0}{#2}
\end{scope}}
\newcommand{\closeLefta}[2]
{\begin{scope}[shift={(#1,0)}]
    \draw[midarrow] (-1,0.5) to[out=90,in=180] (0,1.5);
    \draw[midarrow] (0,-1.5) to[out=180,in=270] (-1,-0.5);
    \mat{-1}{0}{#2}
\end{scope}}

\newcommand{\closeRight}[2]{
    \draw (#1,+1.5) to[out=0,in=90] (#1+1,0.5);
    \draw (#1+1,-0.5) to[out=270,in=0] (#1,-1.5);
    \mat{#1+1}{0}{#2}
}
\newcommand{\closeRighta}[2]{
    \draw[midarrow] (#1+1,0.5) to[out=90,in=0] (#1,+1.5);
    \draw[midarrow] (#1,-1.5) to[out=0,in=-90] (#1+1,-0.5);
    \mat{#1+1}{0}{#2}
}

\newcommand{\colmat}[3]{
    \tensor{#1}{1.5}{#2}
    \tensor{#1}{-1.5}{#3}
    \lineV{-1}{1}{#1}
}
\newcommand{\colmata}[3]{
    \tensor{#1}{1.5}{#2}
    \tensor{#1}{-1.5}{#3}
    \lineVa{-1}{1}{#1}
}

\newcommand{\colmatL}[3]{
    \tensorL{#1}{1.5}{#2}
    \tensorL{#1}{-1.5}{#3}
    \lineV{-1}{1}{#1}
}
\newcommand{\colmatLa}[3]{
    \tensorL{#1}{1.5}{#2}
    \tensorL{#1}{-1.5}{#3}
    \lineVa{-1}{1}{#1}
}

\newcommand{\colmatR}[3]{
    \tensorR{#1}{1.5}{#2}
    \tensorR{#1}{-1.5}{#3}
    \lineV{-1}{1}{#1}
}
\newcommand{\colmatRa}[3]{
    \tensorR{#1}{1.5}{#2}
    \tensorR{#1}{-1.5}{#3}
    \lineVa{-1}{1}{#1}
}

\newcommand{\applyTransferL}[3]{
    \begin{diagram}
        \dobase{0}{0}
        \colmat{0}{#1}{#3} \closeLeft{-0.5}{#2}
        \lineH{0.5}{1}{1.5} \lineH{0.5}{1}{-1.5}
    \end{diagram}
}
\newcommand{\applyTransferLa}[3]{
    \begin{diagram}
        \dobase{0}{0}
        \colmata{0}{#1}{#3} 
        \closeLefta{-0.5}{#2}
        \lineHa{0.5}{1}{1.5} 
        \lineHa{1}{0.5}{-1.5}
    \end{diagram}
}

\newcommand{\applyTransferR}[3]{
    \begin{diagram}
        \dobase{0}{0}
        \colmat{0}{#1}{#3} \closeRight{0.5}{#2}
        \lineH{-0.5}{-1}{1.5} \lineH{-0.5}{-1}{-1.5}
    \end{diagram}
}

\newcommand{\lrtimesMPS}[3]{
    \begin{diagram}
        \dobase{0}{0}
        \lineH{-3}{-2.5}{0}
        \mat{-2}{0}{#1}
        \lineH{-1.5}{-0.5}{0}
        \tensor{0}{0}{#2}
        \lineV{-1}{-0.5}{0}
        \lineH{0.5}{1.5}{0}
        \mat{2}{0}{#3}
        \lineH{2.5}{3}{0}
    \end{diagram}
}
\newcommand{\lrtimesMPSa}[3]{
    \begin{diagram}
        \dobase{0}{0}
        \lineHa{-3}{-2.5}{0}
        \mat{-2}{0}{#1}
        \lineHa{-1.5}{-0.5}{0}
        \tensor{0}{0}{#2}
        \lineVa{-1}{-0.5}{0}
        \lineHa{0.5}{1.5}{0}
        \mat{2}{0}{#3}
        \lineHa{2.5}{3}{0}
    \end{diagram}
}

\newcommand{\ltimesMPS}[2]{
    \begin{diagram}
        \dobase{0}{0}
        \mat{0}{0}{#1}
        \tensor{2}{0}{#2}
        \lineH{-1}{-0.5}{0}
        \lineH{0.5}{1.5}{0}
        \lineH{2.5}{3}{0}
        \lineV{-1}{-0.5}{2}
    \end{diagram}
}
\newcommand{\ltimesMPSa}[2]{
    \begin{diagram}
        \dobase{0}{0}
        \mat{0}{0}{#1}
        \tensor{2}{0}{#2}
        \lineHa{-1}{-0.5}{0}
        \lineHa{0.5}{1.5}{0}
        \lineHa{2.5}{3}{0}
        \lineVa{-1}{-0.5}{2}
    \end{diagram}
}

\newcommand{\rtimesMPS}[2]{
    \begin{diagram}
        \dobase{0}{0}
        \tensor{0}{0}{#1}
        \mat{2}{0}{#2}
        \lineH{-1}{-0.5}{0}
        \lineH{0.5}{1.5}{0}
        \lineH{2.5}{3}{0}
        \lineV{-1}{-0.5}{0}
    \end{diagram}
}
\newcommand{\rtimesMPSa}[2]{
    \begin{diagram}
        \dobase{0}{0}
        \tensor{0}{0}{#1}
        \mat{2}{0}{#2}
        \lineHa{-1}{-0.5}{0}
        \lineHa{0.5}{1.5}{0}
        \lineHa{2.5}{3}{0}
        \lineVa{-1}{-0.5}{0}
    \end{diagram}
}

% --------------------------------
% Special math fonts and notations
% --------------------------------

\newcommand{\graphics}[2]{\includegraphics[totalheight={#1}]{#2}}
\newcommand{\tr}{\operatorname{tr}}
\newcommand{\Abs}[1]{
    \left\lVert #1 \right\rVert
}
\newcommand{\bs}[1]{\boldsymbol{#1}}
\newcommand{\p}{\mathsf{P}}
\newcommand{\h}[1]{\ensuremath{\hat{#1}}}
\renewcommand{\t}[1]{\ensuremath{\tilde{#1}}}
\newcommand{\Z}{\ensuremath{\mathbb{Z}}}
\newcommand{\C}{\ensuremath{\mathbb{C}}}
\newcommand{\R}{\ensuremath{\mathbb{R}}}
\newcommand{\voperator}[1]{\ensuremath{\breve{#1}}}
\newcommand{\manifold}[1]{\ensuremath{\mathcal{#1}}}
\newcommand{\group}[1]{\ensuremath{\mathsf{#1}}}
\newcommand{\vectorspace}[1]{\ensuremath{\mathbb{#1}}}
\newcommand{\algebra}[1]{\ensuremath{\mathfrak{#1}}}
\newcommand{\pf}{\operatorname{pf}}

\newcommand{\Imag}{\mathrm{Im}}
\newcommand{\Real}{\mathrm{Re}}
\newcommand{\sgn}{\mathrm{sgn}}
\newcommand{\T}{\mathsf{T}}
\newcommand{\diag}{\mathrm{diag}}

\newcommand{\e}{\ensuremath{\mathrm{e}}}
\renewcommand{\d}{\ensuremath{\mathrm{d}}}
\renewcommand{\dag}{^\dagger}
\newcommand{\one}{\mathds{1}}
\newcommand{\ol}[1]{\overline{#1}}

\newcommand{\bA}{\bar{A}}
\newcommand{\tA}{\tilde{A}}
\newcommand{\bB}{\bar{B}}
\newcommand{\bG}{\bar{G}}
\newcommand{\tC}{\tilde{C}}

\newcommand{\vect}[1]{\ensuremath{\boldsymbol{#1}}}
\newcommand{\lv}{\ensuremath{\vect{v}_L^\dagger}}
\newcommand{\rv}{\ensuremath{\vect{v}_R}}
\newcommand{\spst}{\ensuremath{\ket{\{s\}}}}
\newcommand{\E}[2]{\ensuremath{E^{#1}_{#2}}}
\renewcommand{\O}[2]{\ensuremath{O^{#1}_{#2}}}
\renewcommand{\H}[4]{\ensuremath{H^{#1#2}_{#3#4}}}
\newcommand{\I}[6]{\ensuremath{I^{#1#2#3}_{#4#5#6}}}
\newcommand{\HH}[6]{\ensuremath{H^{#1#2#3}_{#4#5#6}}}
\newcommand{\J}[4]{\ensuremath{J^{#1#2}_{#3#4}}}
\newcommand{\alphap}{\ensuremath{{\alpha'}}}
\newcommand{\betap}{\ensuremath{{\beta'}}}
\newcommand{\gammap}{\ensuremath{{\gamma'}}}

% group theory
\newcommand{\U}{\ensuremath{\mathrm{U}}}
\newcommand{\SU}{\ensuremath{\mathrm{SU}}}
\newcommand{\su}{\ensuremath{\mathfrak{su}}}
\newcommand{\SO}{\ensuremath{\mathrm{SO}}}
\newcommand{\so}{\ensuremath{\mathfrak{so}}}

% category theory
\newcommand{\Hom}{\ensuremath{\mathrm{Hom}}}

\graphicspath{{./images/}}

\def\cirrad{0.1}
\def\pad{0.1}
\def\shift{0.5}
\newcommand{\drawTbase}[5]{
    \draw[#5] (#1,#2) to (#1+#4,#2);
    \draw[#5] (#1-#4,#2) to (#1,#2);
    \draw[#5] (#1,#2) to (#1,#2+#4);
    \draw[#5] (#1,#2-#4) to (#1,#2);
    \fill[#3] (#1,#2) circle (\cirrad);
}
\newcommand{\drawTket}[3]{
    \draw[midarrow] (#1+\shift,#2+\shift) -- (#1,#2);
    \drawTbase{#1}{#2}{#3}{1}{midarrow}
}
\newcommand{\drawTbra}[3]{
    \draw[midarrow] (#1,#2) -- (#1-\shift,#2-\shift);
    \drawTbase{#1}{#2}{#3}{1}{midarrowrev}
}
\newcommand{\drawXket}[3]{
    \drawTbase{#1}{#2}{#3}{1}{midarrow}
}
\newcommand{\drawXbra}[3]{
    \drawTbase{#1}{#2}{#3}{1}{midarrowrev}
}
\newcommand{\drawaR}[3]{
    \draw[midarrow] (#1-1,#2) to (#1,#2);
    \draw[midarrow] (#1,#2) to (#1+1,#2);
    \draw[midarrow] (#1+\shift,#2+\shift) to (#1,#2);
    \draw[#3, fill=white] (#1,#2) circle (\cirrad);
}
\newcommand{\drawgate}[2]{
    \begin{scope}[shift={(#1,#2)}, xslant=1]
        \draw[fill=orange] 
        (-0.75,-0.2) rectangle ++(1.5,0.4);
        \draw (0,0) node {$g_{ij}$};
        \foreach \x in {-0.5,0.5}{
            \draw[midarrow] (\x,0.2+\shift) -- +(0,-\shift);
            \draw[midarrow] (\x,-0.2) -- +(0,-\shift);
        }
    \end{scope}
}

\NewDocumentCommand{\tensora}{m m O{green}}{
    \begin{scope}[shift={(#1,#2)}]
        \draw[midarrow] (0,0) -- +(1,0);
        \draw[midarrow] (-1,0) -- +(1,0);
        \draw[midarrow] (0,0) -- +(0,1);
        \draw[midarrow] (0,-1) -- +(0,1);
        \draw[fill=#3] (0,0) circle (\pad);
    \end{scope}
}
\NewDocumentCommand{\redmat}{m m O{green}}{
    \begin{scope}[shift={(#1,#2)}]
        \draw[midarrow] (0,0) -- +(1,0);
        \draw[midarrow] (-1,0) -- +(1,0);
        \draw[midarrow] (0,0) -- +(0,1);
        \draw[midarrow] (0,-1) -- +(0,1);
        \draw[fill=#3] (0,0) circle (\pad);
    \end{scope}
}
\NewDocumentCommand{\ctmnw}{m m O{$C^1$}}{
    \draw[midarrow] (#1,#2) -- +(1,0);
    \draw[midarrow] (#1,#2-1) -- +(0,1);
    \node[anchor=south] at (#1,#2+0.1) {#3};
    \draw[fill=orange] (#1-\pad,#2-\pad) rectangle (#1+\pad,#2+\pad);
}
\NewDocumentCommand{\ctmne}{m m O{$C^2$}}{
    \draw[midarrow] (#1-1,#2) -- +(1,0);
    \draw[midarrow] (#1,#2-1) -- +(0,1);
    \node[anchor=south] at (#1,#2+0.1) {#3};
    \draw[fill=orange] (#1-\pad,#2-\pad) rectangle (#1+\pad,#2+\pad);
}
\NewDocumentCommand{\ctmsw}{m m O{$C^4$}}{
    \draw[midarrow] (#1,#2) -- +(1,0);
    \draw[midarrow] (#1,#2) -- +(0,1);
    \node[anchor=north] at (#1,#2-0.1) {#3};
    \draw[fill=orange] (#1-\pad,#2-\pad) rectangle (#1+\pad,#2+\pad);
}
\NewDocumentCommand{\ctmse}{m m O{$C^3$}}{
    \draw[midarrow] (#1-1,#2) -- +(1,0);
    \draw[midarrow] (#1,#2) -- +(0,1);
    \node[anchor=north] at (#1,#2-0.1) {#3};
    \draw[fill=orange] (#1-\pad,#2-\pad) rectangle (#1+\pad,#2+\pad);
}

\NewDocumentCommand{\ctmn}{m m O{$T^1$}}{
    \begin{scope}[shift={(#1,#2)}]
        \draw[midarrow] (-1,0) -- +(1,0);
        \draw[midarrow] (0,0) -- +(1,0);
        \draw[midarrow] (0,-1) -- +(0,1);
        \draw[midarrow, opacity=0.4] (\shift,\shift-1) -- +(0,1);
        \draw (0,0) -- +(\shift,\shift);
        \node[anchor=south] at (0,0+0.1) {#3};
        \draw[fill=yellow] (0,0) circle (\pad);
    \end{scope}
}
\NewDocumentCommand{\ctme}{m m O{$T^2$}}{
    \begin{scope}[shift={(#1,#2)}]
        \draw[midarrow] (0,-1) -- +(0,1);
        \draw[midarrow] (0,0) -- +(0,1);
        \draw[midarrow] (-1,0) -- +(1,0);
        \draw[midarrow, opacity=0.4] (\shift-1,\shift) -- +(1,0);
        \draw (0,0) -- +(\shift,\shift);
        \node[anchor=west] at (0.1,0) {#3};
        \draw[fill=yellow] (0,0) circle (\pad);
    \end{scope}
}
\NewDocumentCommand{\ctms}{m m O{$T^3$}}{
    \begin{scope}[shift={(#1,#2)}]
        \draw[midarrow] (-1,0) -- +(1,0);
        \draw[midarrow] (0,0) -- +(1,0);
        \draw[midarrow] (0,0) -- +(0,1);
        \draw[midarrow, opacity=0.4] (\shift,\shift) -- +(0,1);
        \draw (0,0) -- +(\shift,\shift);
        \node[anchor=north] at (0,-0.1) {#3};
        \draw[fill=yellow] (0,0) circle (\pad);
    \end{scope}
}
\NewDocumentCommand{\ctmw}{m m O{$T^4$}}{
    \begin{scope}[shift={(#1,#2)}]
        \draw[midarrow] (0,-1) -- +(0,1);
        \draw[midarrow] (0,0) -- +(0,1);
        \draw[midarrow] (0,0) -- +(1,0);
        \draw[midarrow, opacity=0.4] (\shift,\shift) -- +(1,0);
        \draw (0,0) -- +(\shift,\shift);
        \node[anchor=east] at (-0.1,0) {#3};
        \draw[fill=yellow] (0,0) circle (\pad);
    \end{scope}
}

\NewDocumentCommand{\projl}{m m O{$P^a$}}{
    \draw[midarrow] (#1-1,#2+0.5) -- ++(1,0);
    \draw[midarrow] (#1-1,#2-0.5) -- ++(1,0);
    \draw[midarrow] (#1+0.3,#2) -- ++(0.7,0);
    \draw[fill=cyan] (#1,#2-0.7) -- (#1+0.3,#2) -- (#1,#2+0.7) -- cycle;
    \node[anchor=south west] at (#1+0.2,#2) {#3};
}
\NewDocumentCommand{\projr}{m m O{$P^b$}}{
    \draw[midarrow] (#1,#2+0.5) -- ++(1,0);
    \draw[midarrow] (#1,#2-0.5) -- ++(1,0);
    \draw[midarrow] (#1-1,#2) -- ++(0.7,0);
    \draw[fill=cyan] (#1,#2-0.7) -- (#1-0.3,#2) -- (#1,#2+0.7) -- cycle;
    \node[anchor=south east] at (#1-0.2,#2) {#3};
}

\begin{document}

\section{Full update of iPEPS}
\label{sec:full-update}

Here we describe the full update improved in Ref. \citenum{Phien2015} when the Hamiltonian is the sum of only \emph{nearest neighbor} terms. For simplicity, we assume the iPEPS is bipartite, generated from only two local tensors:
\begin{equation}
    \ket{\Psi} = \begin{diagram}[0.9][0.9]
        \dobase{0}{0.5} 
        \foreach \x in {0,1} \foreach \y in {0,1}
        {\draw[midarrow] (\x+0.3,\y+0.3) -- (\x,\y);}
        \tensora{0}{0}[blue] \tensora{1}{1}[blue]
        \tensora{0}{1}[green] \tensora{1}{0}[green]
        \node[anchor=north east] at (0,0) {$A$};
        \node[anchor=north east] at (1,0) {$B$};
    \end{diagram}, \qquad
    T_{x,y} = T_{x+1,y+1} = \begin{cases}
        A & x+y = 0 \ \mathrm{mod} \ 2, \\
        B & x+y = 1 \ \mathrm{mod} \ 2.
    \end{cases}
\end{equation}

\subsection{Applying gate on reduced tensors}

Suppose we apply a nearest neighbor gate $g_{ij}$ on a horizontal bond $i = (x,y), j = (x+1,y)$. To reduce computation cost, the gate is applied in the following steps.
\begin{itemize}
    \item Use QR/LQ decomposition to decompose $A, B$:
    \begin{equation}
        \begin{diagram}[0.9][0.9]
            \dobase{0}{0}
            \drawTket{0}{0}{blue}
            \node[anchor=north east] at (0,0) {$A$};
        \end{diagram} = \begin{diagram}[0.9][0.9]
            \dobase{0}{0}
            \drawXket{0}{0}{blue} \drawaR{1}{0}{blue};
            \node[anchor=north east] at (0,0) {$X$};
            \node[anchor=north] at (1,0) {$a_R$};
        \end{diagram}, 
        \quad
        \begin{diagram}[0.9][0.9]
            \dobase{0}{0}
            \drawTket{0}{0}{green}
            \node[anchor=north east] at (0,0) {$B$};
        \end{diagram} = \begin{diagram}[0.9][0.9]
            \dobase{0}{0}
            \drawaR{-1}{0}{green} \drawXket{0}{0}{green} 
            \node[anchor=north west] at (0,0) {$Y$};
            \node[anchor=north] at (-1,0) {$b_L$};
        \end{diagram}.
    \end{equation}

    \item The gate $g_{ij}$ now acts on the 3-leg tensors $a_R$ and $b_L$:
    \begin{equation}
        \begin{diagram}[0.9][0.9]
            \dobase{0}{0}
            \drawTket{0}{0}{blue}
            \drawTket{1}{0}{green}
            \node[anchor=north east] at (0,0) {$A$};
            \node[anchor=north west] at (1,0) {$B$};
            \drawgate{1.2}{0.7}
        \end{diagram} = \begin{diagram}[0.9][0.9]
            \dobase{0}{0}
            \drawXket{-1}{0}{blue} \drawaR{0}{0}{blue}
            \drawaR{1}{0}{green} \drawXket{2}{0}{green}
            \node[anchor=north east] at (-1,0) {$X$};
            \node[anchor=north] at (0,0) {$a_R$};
            \node[anchor=north] at (1,0) {$b_L$};
            \node[anchor=north west] at (1.5,0) {$Y$};
            \drawgate{1.2}{0.7}
        \end{diagram}. 
    \end{equation}
    
    \item Perform SVD, obtaining the updated 3-leg tensors $a'_R$, and $b'_L$:
    \begin{equation}
        \begin{diagram}[0.9][0.9]
            \dobase{0}{0}
            \drawaR{0}{0}{blue} \drawaR{1}{0}{green}
            \node[anchor=north] at (0,0) {$a_R$};
            \node[anchor=north] at (1,0) {$b_L$};
            \drawgate{1.2}{0.7}
            \draw[dashed, red] (0.2,-0.3) -- ++(1.5,1.5);
        \end{diagram}
        = \begin{diagram}[0.9][0.9]
            \dobase{0}{0}
            \drawaR{0}{0}{blue} \drawaR{1}{0}{green}
            \node[anchor=north] at (0,0) {$a'_R$};
            \node[anchor=north] at (1,0) {$b'_L$};
        \end{diagram}. 
    \end{equation}
\end{itemize}

\subsection{Fast full update}

The bond dimension between $a'_R, b'_L$ is truncated by minimizing the cost function
\begin{equation}
    f(\tilde{a}_R, \tilde{b}_L) = \lVert 
        \ket{\Psi_{\tilde{a}_R \tilde{b}_L}} 
        - \ket{\Psi_{a'_R b'_L}}
    \rVert^2,
\end{equation}
where $\ket{\Psi_{a'_R b'_L}} = g_{(x,y)(x+1,y)} \ket{\Psi}$, and $\ket{\Psi_{\tilde{a}_R \tilde{b}_L}}$ is obtained by replacing $a'_R, b'_L$ with $\tilde{a}_R, \tilde{b}_L$, which have a smaller bond dimension. The minimization is done iteratively as follows:
\begingroup
\newcommand{\backbone}{
    \draw (3,0) -- +(2,0);
    \draw (3,4) -- +(2,0);
    \ctmsw{0}{0} \ctmse{8}{0} 
    \ctmnw{0}{4} \ctmne{8}{4} 
    \ctmn{2}{4} \ctmn{6}{4} \ctme{8}{2} 
    \ctms{2}{0} \ctms{6}{0} \ctmw{0}{2}
    \tensora{2}{2}[blue]
    \node[anchor=north east] at (2,2) {$X$};
    \tensora{6}{2}
    \node[anchor=north east] at (6,2) {$Y$};
    \begin{scope}[shift={(\shift,\shift)}, opacity=0.4]
        \tensora{2}{2}[blue]
        \node[anchor=south west] at (2,2) {$X^\dagger$};
        \tensora{6}{2}
        \node[anchor=south west] at (6,2) {$Y^\dagger$};
    \end{scope}
}
\begin{itemize}
    \item We first fix $\tilde{b}_L$ and optimize $\tilde{a}_R$. The cost function can be expressed as a bilinear of $\tilde{a}_R$:
    \begin{align}
        f_a(\tilde{a}_R, \tilde{a}^\dagger_R)
        &= \braket{\Psi_{\tilde{a}_R \tilde{b}_L} | \Psi_{\tilde{a}_R \tilde{b}_L}}
        - \braket{\Psi_{\tilde{a}_R \tilde{b}_L} | \Psi_{a'_R b'_L}}
        - \braket{\Psi_{a'_R b'_L} | \Psi_{\tilde{a}_R \tilde{b}_L}}
        + \braket{\Psi_{a'_R b'_L} | \Psi_{a'_R b'_L}}
        \nonumber \\
        &= \tilde{a}^\dagger_R R_a \tilde{a}_R
        - \tilde{a}^\dagger_R S_a
        - S^\dagger_a \tilde{a}_R + T
    \end{align}
    which is minimized with respect to $\tilde{a}_R$ by
    \begin{equation}
        \partial f / \partial \tilde{a}^\dagger_R = 0
        \ \Rightarrow \ 
        R_a \tilde{a}_R = S_a. 
    \end{equation}

    \item Using the CTMs of $\ket{\Psi}$, the tensors $R_a, S_a, T$ can be expressed as
    \begin{equation}
    \begin{gathered}
        R_a = \begin{diagram}[0.8][0.6]
            \dobase{0}{2} \backbone
            \draw[green] (5,2) circle (\cirrad);
            \draw[midarrow] (4,2) -- (5,2);
            \node[anchor=north] at (5,2) {$\tilde{b}_L$};
            \begin{scope}[shift={(\shift,\shift)}, opacity=0.4]
                \draw[green] (5,2) circle (\cirrad);
                \draw[midarrow] (4,2) -- (5,2);
                \node[anchor=south] at (5,2) {$\tilde{b}^\dagger_L$};
            \end{scope}
            \draw[midarrow] (5+\shift,2+\shift) -- (5,2);
        \end{diagram}, \ \ 
        S_a = \begin{diagram}[0.8][0.6]
            \dobase{0}{2} \backbone
            \draw[green] (5,2) circle (\cirrad);
            \draw[blue] (4-\shift,2) circle (\cirrad);
            \draw[midarrow] (4,2+\shift) -- +(-\shift,-\shift);
            \draw (3,2) -- (4-\shift,2);
            \draw[midarrow] (4-\shift,2) -- (5,2);
            \node[anchor=north] at (4-\shift,2) {$a'_R$};
            \node[anchor=north] at (5,2) {$b'_L$};
            \begin{scope}[shift={(\shift,\shift)}, opacity=0.4]
                \draw[green] (5,2) circle (\cirrad);
                \draw[midarrow] (4,2) -- (5,2);
                \node[anchor=south] at (5,2) {$\tilde{b}^\dagger_L$};
            \end{scope}
            \draw[midarrow] (5+\shift,2+\shift) -- (5,2);
        \end{diagram}
        \\~\\
        T = \begin{diagram}[0.8][0.6]
            \dobase{0}{2} \backbone
            \draw[blue] (3,2) circle (\cirrad);
            \draw[green] (5,2) circle (\cirrad);
            \draw[midarrow] (3,2) -- (5,2);
            \node[anchor=north] at (3,2) {$a'_R$};
            \node[anchor=north] at (5,2) {$b'_L$};
            \begin{scope}[shift={(\shift,\shift)}, opacity=0.4]
                \draw[blue] (3,2) circle (\cirrad);
                \draw[green] (5,2) circle (\cirrad);
                \draw[midarrow] (3,2) -- (5,2);
                \node[anchor=south west] at (3,2) {$a'^\dagger_R$};
                \node[anchor=south] at (5,2) {$b'^\dagger_L$};
            \end{scope}
            \draw[midarrow] (5+\shift,2+\shift) -- (5,2);
            \draw[midarrow] (3+\shift,2+\shift) -- (3,2);
        \end{diagram}, \quad \text{where} \quad
        \begin{aligned}
            A' &= X a'_R, &
            B' &= b'_L Y
            \\
            \tilde{A} &= X \tilde{a}_R, &
            \tilde{B} &= \tilde{b}_L Y
        \end{aligned}
    \end{gathered}
    \end{equation}
    Here the conjugated tensors have been modified by the flippers to have the same arrow direction as the original tensors. The numbers on each bond labels the \texttt{ncon} indices in program implementation. 

    \item After solving for $\tilde{a}_R$, we fix $\tilde{a}_R$ and optimize $\tilde{b}_L$. The cost function is expressed as a bilinear of $\tilde{b}_L$:
    \begin{equation}
        f_b(\tilde{b}_L, \tilde{b}^\dagger_L)
        = \tilde{b}^\dagger_L R_b \tilde{b}_L
        - \tilde{b}^\dagger_L S_b
        - S^\dagger_b \tilde{b}_L + T,
    \end{equation}
    which is minimized by 
    \begin{equation}
        R_b \tilde{b}_L = S_b.
    \end{equation}
    The tensors $R_b, S_b$ are given by
    \begin{equation}
        R_b = \begin{diagram}[0.8][0.6]
            \dobase{0}{2} \backbone
            \draw[blue] (3,2) circle (\cirrad);
            \draw[midarrow] (3,2) -- (4,2);
            \node[anchor=north] at (3,2) {$\tilde{a}_R$};
            \begin{scope}[shift={(\shift,\shift)}, opacity=0.4]
                \draw[blue] (3,2) circle (\cirrad);
                \draw[midarrow] (3,2) -- (4,2);
                \node[anchor=south west] at (3,2) {$\tilde{a}^\dagger_R$};
            \end{scope}
            \draw[midarrow] (3+\shift,2+\shift) -- (3,2);
        \end{diagram}, \ \ 
        S_b = \begin{diagram}[0.8][0.6]
            \dobase{0}{2} \backbone
            \draw[blue] (3,2) circle (\cirrad);
            \draw[green] (4+\shift,2) circle (\cirrad);
            \draw[midarrow] (3,2) -- (4+\shift,2);
            \draw (4+\shift,2) -- (5,2);
            \node[anchor=north] at (3,2) {$a'_R$};
            \node[anchor=north] at (4+\shift,2) {$b'_L$};
            \begin{scope}[shift={(\shift,\shift)}, opacity=0.4]
                \draw[blue] (3,2) circle (\cirrad);
                \draw[midarrow] (3,2) -- (4,2);
                \node[anchor=south west] at (3,2) {$\tilde{a}^\dagger_R$};
            \end{scope}
            \draw[midarrow] (3+\shift,2+\shift) -- (3,2);
            \draw[midarrow] (5,2+\shift) -- (4+\shift,2);
        \end{diagram}
    \end{equation}

    \item When then iteratively solve for $\tilde{a}_R, \tilde{b}_L$ until the cost function $f_a, f_b$ is sufficiently small. Then all $A, B$ tensors in the entire iPEPS are replaced by $\tilde{A}, \tilde{B}$.
\end{itemize}
\endgroup

\subsection{Generalization to arbitrary unit cell}

\bibliographystyle{ieeetr}
\bibliography{./refs}

\end{document}
